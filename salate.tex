%\rezept{Titel}{20}{Untertitel}{150 & ml & Ei \\3 & Laib & Brot }{Ei aufs Brot spiegeln und mit Seffer und Pfalz bestreuen.}

\rezept{Bohnensalat}{20}{
a la Mama}{
1 & Becher & saure Sahne \\
1 & & Zwiebel \\
1 & Zehe & Knoblauch \\
& & Bohnenkraut \\ 
& & Salz \\
& & Pfeffer \\
& & Zucker \\
& & Essig \\
& und & \\
& & gr\"une Bohnen
}{ Bohnen kochen und warm in die So\ss{}e geben.}

\rezept{Brotsalat}{20}{Verwertung f\"r altes, hartgewordenes Brot}{
 & & Feldsalat \\
 & oder & Rucola \\
 & etwas & altes Brot \\
1 & & Paprika \\
1/2 & & Gurke \\
 & & (Cocktail)tomaten \\
 & & (Walnuss)\"ol
}{
Gem\"use und Brot in kleine W\"urfel schneiden, Salat putzen.
Alles in eine Sch\"ussel geben, mit \"Ol betr\"aufeln, salzen und pfeffern.
Ziehen lassen - fertig.
}

\newpage

\rezept{Kartoffelsalat}{20}{
mit Grillkartoffeln und Grillgem\"use, 4-6 Personen, 20 min}{
& & Pistazienpesto \\
1 & kg & mittelgro\ss{}e \\
& & festkochende Kartoffeln \\
& & grobes Meersalz \\
2 & & grosse Paprika \\
2 & EL & Oliven\"ol \\
& & Pfeffer \\
2 & EL & frisches Basilikum
}{
Die Kartoffeln waschen, achteln, in einen gro\ss{}en Topf geben und gut mit Wasser bedecken.
2 TL Salz hinzuf\"ugen und aufkochen.
Die Hitze reduzieren und die Kartoffeln 5-10 Min kochen, bis sie knapp gar sind.
In der Zwischenzeit die Paprikaschoten l\"angs halbieren und in 3 cm breite St\"ucke schneiden.

Die Kartoffeln abgie\ss{}en und zur\"uck in den leeren Topf geben. 
Die Paprikast\"ucke sowie 2 EL \"Ol und 1/2 TL Salz hinzuf\"ugen und alles gut vermischen.

Den Grill f\"ur mittlere Hitze vorheizen. 
Sobald sie hei\ss{} ist, Kartoffeln und Paprika m\"oglichst flach auf einem mit Backpapier ausgelegten Blech verteilen.
Etwa 10 - 15 Min grillen bis die Kartoffeln auf allen Seiten knusprig braun und innen weich sind.
Gelegentlich wenden.
Das Gem\"use in eine Sch\"ussel mit dem Pistazienpesto geben. 
Vorsichtig mischen bis das Gem\"use \"uberall mit Pesto bedeckt ist.
Mindestens 5 Minuten abk\"ulen lassen und mit Salz und Pfeffer abschmecken.

Warm oder auf Zimmertemperatur abgek\"ult servieren.
}

\rezept{Linsensalat}{20}{
Suchen noch das beste Rezept}{
200 & g & Linsen (Teller- oder Beluga) \\
}{
Wir brauchen ein Rezept!
}
