%\rezept{Titel}{20}{Untertitel}{150 & ml & Ei \\3 & Laib & Brot }{Ei aufs Brot spiegeln und mit Seffer und Pfalz bestreuen.}

\rezept{Linsensalat}{20}{}{
1 & & Zwiebel \\
250 & g & Belugalinsen \\
1/2 & l & Gemüsebrühe \\
1-2 & Blatt & Lorbeer \\
2 & & Schalotten \\
2-3 & & Möhren \\
1 & Zehe & Knoblauch \\
1 & Bund & frischer Majoran \\
& & oder 1 TL trockenen Oregano \\
6 & EL & Essig (Balsamico, Himbeer oder Weißwein) \\
1 & EL & scharfer Senf \\
& & Pfeffer \\
1 & Prise & brauner Zucker \\
1/8 & l & Olivenöl \\
& & Schafskäse
}{
Zwiebel schälen und halbieren.
Möhren putzen und in kleine Würfel schneiden.
Linsen, Zwiebel, Lorbeerblätter in der Geüsebrühe aufkochen und die Linsen gar (noch kernig) kochen.
Kurz vor Schluss die Möhren dazugeben (sie sollen noch knackig sein).

Schalotten und Knoblauch sehr fein hacken.

Essig, Senf, Zucker verrühren.
Olivenöl kräftig darunter schlagen.
Knoblauch, Schalotten, Majoran daruntermischen.

Zwiebel und Lorbeer aus den Linsen nehmen.

Linsen in der Garflüssigkeit abkühlen lassen, lauwarm mit der Vinaigrette vermischen und ca. 3 Stunden ziehen lassen.

Den Schafskäse würfeln und zum Salat reichen.
}

\rezept{Bohnensalat}{20}{
a la Mama}{
1 & Becher & saure Sahne \\
1 & & Zwiebel \\
1 & Zehe & Knoblauch \\
& & gr\"une Bohnen
}{
Für die Soße die saure Sahne mit Salz, Pfeffer, Zucker, Essig und geriebenen Knoblauch vermischen.

Bohnen mit Bohnenkraut und Zwiebeln kochen und warm in die Soße geben. }

\rezept{Coleslaw}{20}{
aka Krautsalat}{
1,5 & kg & Weißkohl \\
450 & g  & Möhren \\
1   &    & Zitronen \\
2   & TL & Salz \\
2   & TL & Zucker \\
50  & ml & Milch \\
30  & ml & Öl \\
50  & g  & Joghurt \\
1   & TL & Essig
}{
Weißkohl und Möhren in feine Streifen schneiden, zum Beispiel mit einer Mandoline und in einer Schüssel mit Salz und Zucker vermischen.

Das Kraut kneten bis die gewünschte Konsistenz erreicht ist und sich am Schüsselboden Wasser sammelt, 20 min ziehen lassen.

Milch mit einem Pürierstab mixen und das Öl in kleinen Portionen nach und nach hinzugeben, Joghurt und Essig dazugeben, pfeffern.

Zitronen auspressen und mit der Soße über den Salat geben, nochmal ziehen lassen.}

\newpage

\rezept{Brotsalat - Panzanella}{20}{Verwertung f\"r altes, hartgewordenes Brot}{
 & & Feldsalat \\
 & oder & Rucola \\
 & etwas & altes Brot \\
1 & & Paprika \\
1/2 & & Gurke \\
 & & (Cocktail)tomaten \\
 & & (Walnuss)\"ol
}{
Gem\"use und Brot in kleine W\"urfel schneiden, Salat putzen.
Alles in eine Sch\"ussel geben, mit \"Ol betr\"aufeln, salzen und pfeffern.
Ziehen lassen - fertig.
}

\rezept{Kartoffelsalat}{20}{mit Grillkartoffeln und Grillgem\"use, 4-6 Personen, 20 min}{
& & Pistazienpesto \\
1 & kg & mittelgro\ss{}e \\
& & festkochende Kartoffeln \\
& & grobes Meersalz \\
2 & & grosse Paprika \\
2 & EL & Oliven\"ol \\
& & Pfeffer \\
2 & EL & frisches Basilikum
}{
Die Kartoffeln waschen, achteln, in einen gro\ss{}en Topf geben und gut mit Wasser bedecken.
2 TL Salz hinzuf\"ugen und aufkochen.
Die Hitze reduzieren und die Kartoffeln 5-10 Min kochen, bis sie knapp gar sind.
In der Zwischenzeit die Paprikaschoten l\"angs halbieren und in 3 cm breite St\"ucke schneiden.

Die Kartoffeln abgie\ss{}en und zur\"uck in den leeren Topf geben.
Die Paprikast\"ucke sowie 2 EL \"Ol und 1/2 TL Salz hinzuf\"ugen und alles gut vermischen.

Den Grill f\"ur mittlere Hitze vorheizen.
Sobald sie hei\ss{} ist, Kartoffeln und Paprika m\"oglichst flach auf einem mit Backpapier ausgelegten Blech verteilen.
Etwa 10 - 15 Min grillen bis die Kartoffeln auf allen Seiten knusprig braun und innen weich sind.
Gelegentlich wenden.
Das Gem\"use in eine Sch\"ussel mit dem Pistazienpesto geben.
Vorsichtig mischen bis das Gem\"use \"uberall mit Pesto bedeckt ist.
Mindestens 5 Minuten abk\"ulen lassen und mit Salz und Pfeffer abschmecken.

Warm oder auf Zimmertemperatur abgek\"ult servieren.
}

\rezept{Obstsalat}{20}{erfrischend}{
& & Obst \\
& & Honig \\
& & Orange \\
& & Limette \\
frische & & Minze
}{
Für das Dressing die Orange und Limette auspressen und mit Honig abschmecken, wer mag kann auch noch Zesten dazugeben.
Das Obst und die Minze klein schneiden und mit dem Dressing bedecken, ziehen lassen.
}
