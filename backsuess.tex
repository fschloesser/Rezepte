%\rezept{Titel}{20}{Untertitel}{150 & ml & Ei \\3 & Laib & Brot }{Ei aufs Brot spiegeln und mit Seffer und Pfalz bestreuen.}

\rezept{Mamas Pfannkuchen}{20}{
Reicht für 6 große}{
3 & & Eier \\
200 & g & Mehl \\
200 & ml & Milch \\
1 & Prise & Salz \\
1 & Prise & Zucker
}{
Zutaten zu einemehr fl\"ussigen Teig zusammen mischen und eine Stunde ziehen lassen.

Pfannkuchen in einer gefetteten Pfanne bei mittlerer Hitze (4-5/9) und mit Deckel ausbacken.
Nach Belieben mit Obst oder Gem\"use belegen, ein halber Apfel pro Pfannkuchen.}

\rezept{American Pancakes}{20}{
etwa 4 mittelgroße ~}{
1 & Tasse & Mehl \\
1 & & Ei \\
2 & TL & Backpulver \\
1 & TL & Natron \\
1/3 & Tasse & Zucker und Vanille \\
1 & Tasse & Buttermilch \\
 & & optional Blaubeeren \\
 & & am Besten frisch
}{
Zuerst die trockenen Zutaten mischen und dann die nassen dazu geben.
Zu einem z\"ahfl\"ussigen Teig vermischen und in Butter von beiden Seiten bei mittlerer Hitze ausbacken.
Bei Bedarf kann man Blaubeeren (in die Pfanne, direkt nach dem Teig) oder Schokoladenchips (in den Teig) dazu geben.
Mit dem umdrehen warten, bis die Ränder gebacken aussehen und der Teig von den aufsteigenden Blasen Löcher bekommt..
}

\newpage

\rezept{Joghurt Muffins}{20}{
etwa 12 Stück a 130 kcal}{
200g & (1 cup) & Joghurt \\
150g & (1 1/2 cup) & Mehl \\
2 Msp. & & Backpulver \\
170g & (1 cup) & Zucker \\
60g & (1/3 cup) & Öl \\
2 & & Eier \\
1 Prise & & Salz \\
  & & Füllung (z.B. Blaubeeren)
}{
Info: 1 cup = 225 ml.

Trockene Zutaten mischen,
Nasse Zutaten mischen, Zucker dazu, Mehl und Backpulver dazu, Beeren/Schokolade dazu, alles zusammen rühren.

Teig in die Muffinformen füllen und bei 180°C etwa 40 bis 50 min backen.}

\rezept{Karotten Muffins}{20}{
etwa 15 Stück a 230 cals}{
& & Füllung \\
226 & g & (8oz) Frischkäse \\
1/2 & Tasse & Puderzucker \\
1 & TL & Vanilleextrakt \\
1 & Prise & Salz \\
& & carrot cake \\
1/3 & Tasse & brauner Zucker \\
2/3 & Tasse & Zucker \\
2/3 & Tasse & Öl \\
2 & große & Eier \\
1 & TL & Vanilleextrakt \\
226g & Möhren (oder etwas mehr) \\
3/4 & TL & Salz \\
1 1/4 & TL & pumpkin spice \\
1 1/4 & Tasse & Mehl \\
1 & TL & Natron
}{
Für die Füllung deren Zutaten vermischen bis sich der Zucker aufgelöst hat.
Die Möhren reiben.
In einer großen Schüssel Zucker, Öl, Eier und Vanille vermischen, dann Karotten, Salz und Pumpkinspice daruner rühren.
Das Mehl mit dem Natron sieben und unterheben.
Den Ofen auf 190 Grad Celcius vorheizen und Muffinformen mit Muffinpapier auslegen.
Etwa 2 EL des Teiges in die Formen und darauf einen EL der Füllung geben (Hier wird noch nicht die gesamte Füllung verbraucht).
Die Muffins mit Teig auffüllen und 18-20 Minuten backen.
Auf dem Rost etwa 10 Minuten auskühlen lassen.
Die Füllung in einen Spritzbeutel geben und damit die Muffins füllen.
Notiz: Pumpkin Spice sind 18 Teile gemahlener Zimt, 4 Teile gemahlener Muskatnuss, 4 Teile gemahlener Ingwer, 3 Teile gemahlener Nelken und 3 Teile gemahlener Piment.
}

\rezept{Apfel Muffins}{20}{
etwa 16 Stück}{
250 & g & Mehl \\
150 & g & Zucker \\
125 & g & Butter \\
2 & & Eier \\
1 & Pkg. & Vanillezucker \\
1/2 & Pkg. & Backpulver \\
250 & ml & Milch \\
2 & & Äpfel (z.B. Booskoop)
}{
Zutaten wie Mehl, Zucker, Margarine, Eier, Vanillezucker, Backpulver und Milch in eine große Schüssel geben und mit dem Mixer durchrühren.

Danach die geschälten, in kleine Stücke geschnittenen Äpfel hinzugeben und in den Teig mischen.

Die Papierförmchen jeweils zu 3/4 mit dem Teig füllen und auf ein Backblech stellen.

Im vorgeheizten Backofen bei 150°C Heißluft ca. 20-25 Minuten backen.}

\rezept{Vegane Apfel Karotte Nuss Muffins}{20}{
24 Muffins a 240 kcal}{
300 & g & Karotten \\
200 & g & Apfel \\
150 & g & Wasser \\
10 & EL & Kokosöl \\
300 & g & Zucker \\
400 & g & geriebene Nüsse \\
300 & g & Mehl \\
2 & EL & Backkakao \\
1 & Pkg & Backpulver \\
1 & Prise & Salz
}{
Karotten und Apfel reiben und mit dem Wasser vermischen.
Kokosöl und Zucker vermischen und zu der Apfel-Karotten-Mischung geben.
Die restlichen Zutaten mischen und darunterheben.

Bei 180 Grad Ober- Unterhitze etwa 20-40 Minuten backen (Kuchen braucht länger als Muffins.)
}

\rezept{Profiteroles/Windbeutel}{20}{
}{
250 & ml & Wasser \\
115 & g & Butter \\
3 & g & Salz (1/2 TL) \\
7 & g & Zucker \\
125 & g & Mehl \\
4 & & Eier \\
}{
Das Wasser in einem Topf auf mittlerer Hitze geben.
Die Butter würfeln und im Wasser schmelzen.
Zucker, Salz und Mehl mit einem Holzlöffel untermischen und den Teig konstant rühren bis sich auf dem Boden des Topfes ein Film bildet.
Zwei bis drei Minuten kochen.
Von der heißen Platte nehmen und drei Minuten auskühlen lassen, dann die Eier nacheinander einzeln dazu geben und verrühren.
Der Teig sollte glatt und nicht klebrig sein und eine Spitze halten können.

Mit einem Sprizbeutel in etwa 3.5 bis 4 cm große Kreise spritzen.
Den Finger in Wasser anfeuchten und die Spitzen herunter drücken, sodass sie nicht verbrennen.
Optional mit Eigelb bestreichen.
Bei 190 Grad Ober-Unterhitze etwa 25 - 30 Minuten backen.

Wichtig: dabei den Ofen nicht öffnen, sie fallen sonst zusammen.

Kann mit etwa 700 ml Crema pastelera gefüllt werden oder mit geschlagener Sahne, vielleicht auch mit Käsesoße.
Füllung entweder von unten hineinspritzen oder aufschneiden und füllen.}
