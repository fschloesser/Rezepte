\rezept{Sauerteigbrot}{20}{Tagesprojekt, 2 Brote}{
Levain & & (100\% Hydration) \\
35 & g & Starter \\
35 & g & Weizenvollkornmehl \\
35 & g & Weizenmehl \\
70 & g & Wasser, Raumtemperatur \\
Teig & & (80\% Hydration) \\
175 & g & Levain \\
800 & g & Mehl (Mischung) \\
75 & g & Vollkornmehl \\
700 & g & Wasser \\
20 & g & Salz
}{
Brot:

 8:00 am: Levain ansetzen, dafür alle Zutaten mischen und 5-6 Stunden bei etwa 25 C mit einem feuchten Handtuch abgedeckt ruhen lassen.

11:30 am: Teig ansetzen, dafür Mehl und 620g Wasser mischen und dann mit einem feuchten Handtuch abgedeckt ruhen lassen.

 1:00 pm: Wenn der Levain fertig ist (gerade dann wenn die Oberfläche anfängt zu fallen) zum Teig geben, hineindrücken und vermischen. Slap-and-fold ohne Mehl bis der Teig nicht mehr so schlimm klebt, dann mit einem feuchten Handtuch abgedeckt 25 min ruhen lassen.
Sauerteig nie kneten, nur falten und schlagen.

 1:30 pm: Mit dem Salz und dem restlichen Wasser mischen und slap-and-fold ohne Mehl.

 1:35 pm: Gehen/Fermentieren lassen bei 25 Celsius.

 1:50 pm: Den Teig in der Schüssel falten: den Teig vom Rand hochziehen und über sich selbst falten, einmal um die Schüssel herum. Nr. 1

 2:05 pm: Falten Nr. 2

 2:20 pm: Falten Nr. 3

 2:50 pm: Falten Nr. 4

 3:20 pm: Falten Nr. 5

 3:50 pm: Falten Nr. 6 und weiterhin ruhen lassen.

 6:00 pm: Auf die ungemehlte Arbeitsfläche geben und mit angefeuchteten Händen und angefeuchtetem Teigschaber In zwei teilen und ohne zu kneten grob rund formen.

 6:20 pm: Unbedeckt aber oberflächlich gemehlt auf der Arbeitsfläche 20 Minuten ruhen lassen. Umdrehen und zusammenfalten und in mit großzügig gemehlten Handtüchern ausgelegten Schalen ruhen lassen mit dem Bauch nach oben. (Im Kühlschrank 12-14 Stunden wenn nötig über Nacht.)

 8:20 am oder am nächsten Morgen: Den ersten Laib backen (Poke test: Wenn der Teig fast gänzlich aber nicht komplett zurück springt hat er fertig geruht.). Dutch Oven bei 260 Celsius vorheizen, Brot mehlen und Dutch Oven mehlen, Laib hineinkippen und leicht asymmetrisch einen halben cm tief einschneiden. Erst 20 Minuten lang bei 260 Celsius dann 20-30 Minuten 230 Celsius ohne Deckel backen Herausholen wenn das Brot eine gute Farbe hat. Danach den zweiten Laib genauso backen, dafür den Topf 15 Minuten aufheizen lassen. (Dutch oven - combo cooker - kombi-topf: Eine Pfanne
 und ein Topf, gusseisern die bei Bedarf zusammenpassen.)
Für die zweiten zwanzig Minuten ein Blech als Hitzeschirm unter den Topf schieben.

Vorm Anschneiden eine Stunde auf einem Rost oder im halboffenen Ofen auskühlen lassen. }

\rezept{Sauerteigstarter}{20}{etwa 170 kcal pro 100g}{
  & & (Roggen-) Vollkornmehl \\
  & & Weizenmehl \\
  & & lauwarmes Wasser (25 C) \\
}{
In einem Halbliterglas mit aufgelegtem Deckel an einem dunklen, etwa 25 Grad warmen Ort lagern.
Das Gewicht des Glases vorher notieren und Mengen mit einer Waage ausmessen.

Tag 1:
150g handwarmes Wasser und
100g Bio-Roggenvollkornmehl mischen bis die Masse keine Klumpen mehr hat. Mit einem lose sitzenden Deckel abdecken und 12 bis 24 Stunden ruhen lassen.

Tag 2+3:
70g Starter,
50g Bio-Roggenvollkornmehl,
50g ungebleichtes Weizenmehl und
115g handwarmes Wasser

Tag 4+5:
70g Starter,
50g Bio-Roggenvollkornmehl,
50g ungebleichtes Weizenmehl und
100g handwarmes Wasser

Tag 6:
50g Starter,
50g Bio-Roggenvollkornmehl,
50g ungebleichtes Weizenmehl und
100g handwarmes Wasser

Tag 7 und alle weiteren:
25g Starter,
50g Bio-Roggenvollkornmehl,
50g ungebleichtes Weizenmehl und
100g handwarmes Wasser

Im Schrank bei etwa 25 Grad lagern: jeden Tag füttern.

Kann im Kühlschrank gelagert werden, dafür 25g Starter mit 100g Mehl und 100ml Wasser füttern und zwei Stunden später in den Kühlschrank stellen.
Gefüttert wird alle 5 bis 7 Tage.

Kann eingefroren werden, dafür ein paar Stunden nach dem Füttern den aktiven Starter einfrieren.

Kann getrocknet werden, dafür auf einem Backpapier ausstreichen und trocknen, zerbröseln und in einem geschlossenen Gefäß lagern.

Tipp: Nicht mehr benötigten Starter in einer geölten Pfanne zu einem Pfannkuchen backen, z.B. mit Sesam und Frühlingszwiebeln oder mit Rosinen und Zucker bestreuen. }
