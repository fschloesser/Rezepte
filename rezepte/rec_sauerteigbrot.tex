\rezept{Sauerteigbrot ultimate}{20}{Tagesprojekt, 2 Brote, 3500 kcal}{
Levain & & (100\% Hydration) \\
35 & g & Starter \\
35 & g & Weizenvollkornmehl \\
35 & g & Weizenmehl \\
70 & g & Wasser, Raumtemperatur \\
Teig & & (80\% Hydration) \\
175 & g & Levain \\
800 & g & Mehl (Mischung) \\
75 & g & Vollkornmehl \\
700 & g & Wasser \\
20 & g & Salz
}{
Brot:

 8:00 am: Levain ansetzen, dafür alle Zutaten mischen und 5-6 Stunden bei etwa 25 C mit einem feuchten Handtuch abgedeckt ruhen lassen.

11:30 am: Teig ansetzen, dafür die Mehle mischen und mit 620g Wasser vermengen. Dann mit einem feuchten Handtuch abgedeckt ruhen lassen.
Hier binden sich die Glutenstraenge, der Teig sollte dabei zu einer weniger klebrigen, elastischeren Masse werden.

 1:00 pm: Wenn der Levain fertig ist (gerade dann wenn die Oberfläche anfängt zu fallen) selbigen zum Teig geben, mit angefeuchteten Fingern Löcher hineindrücken und alles vermischen.
Den Teig mit der Slap-and-fold Technik ohne Mehl bearbeiten bis er nicht mehr so schlimm klebt, dann mit einem feuchten Handtuch abgedeckt 25 min ruhen lassen.
Sauerteig nie kneten, nur falten und schlagen.

Tip: Der Teig klebt weniger an den Haenden wenn man diese vorher ein wenig anfeuchtet.

 1:30 pm: Mit dem Salz und dem restlichen Wasser vermengen und wieder ohne Mehl slap-and-folden.

(Das Salz nicht vorher dazu geben, da wir dem Mehl die Chance geben wollen sich voll Wasser zu saugen.
Der osmotische Effekt vom Salz würde dies stören.)

 1:35 pm: Gehen/Fermentieren lassen bei 25 Celsius.

 1:50 pm: Den Teig in der Schüssel falten: den Teig vom Rand hochziehen und über sich selbst falten, einmal um die Schüssel herum. Nr. 1

 2:05 pm: Falten Nr. 2

 2:20 pm: Falten Nr. 3

 2:50 pm: Falten Nr. 4

 3:20 pm: Falten Nr. 5

 3:50 pm: Falten Nr. 6 und weiterhin ruhen lassen.

 6:00 pm: Auf die ungemehlte Arbeitsfläche geben und mit angefeuchteten Händen und angefeuchtetem Teigschaber in Zwei teilen und ohne zu kneten grob rund formen.

Tip: Der Teig klebt nicht so schlimm am Teigschaber wenn man ihn schnell wegzieht.

Unbedeckt aber oberflächlich gemehlt auf der Arbeitsfläche 20 Minuten ruhen lassen.

 6:20 pm:
Umdrehen und zusammenfalten und mit dem Teigschaber etwas Oberflaechenspannung erzeugen.
In mit großzügig gemehlten Handtüchern ausgelegten Schalen ruhen lassen mit dem Bauch nach oben. (Im Kühlschrank 12-14 Stunden wenn nötig über Nacht.)

 8:20 oder am nächsten Morgen: Den ersten Laib backen.

Poke test: Wenn der Teig fast gänzlich aber nicht komplett zurück springt hat er fertig geruht.
Wenn er ganz zurück springt braucht er mehr Zeit, wenn er gar nicht zurück springt ist es zu spaet.

Dutch Oven bei 260 Celsius vorheizen, Brot mehlen und Dutch Oven mehlen, Laib hineinkippen und leicht asymmetrisch einen halben cm tief einschneiden.
Erst 20 Minuten lang bei 260 Celsius dann 20-30 Minuten 230 Celsius ohne Deckel backen.
Herausholen wenn das Brot eine gute Farbe hat.
Danach den zweiten Laib genauso backen, dafür den Topf 15 Minuten aufheizen lassen.
(Dutch oven - combo cooker - kombi-topf:
Eine Pfanne und ein Topf, gusseisern die bei Bedarf zusammenpassen.)

Für den zweiten Laib zwanzig Minuten ein Blech als Hitzeschirm unter den Topf schieben.

Vorm Anschneiden eine Stunde auf einem Rost oder im halboffenen Ofen auskühlen lassen. }

\rezept{Sauerteigbrot easy}{20}{1 Brot}{
50 & g & aktiven Sauerteigstarter \\
350 & g & lauwarmes Wasser \\
350 & g & Weizenmehl \\
100 & g & Vollkornmehl \\
10 & g & Salz
}{
Das Wasser mit dem Starter verrühren und danach mit dem Mehl und Salz soweit vermischen, dass keine Klümpchen mehr existieren.
Über Nacht bei etwa 21 Grad stehen lassen.
Wenn der Teig auf etwa das doppelte gewachsen ist und unter schütteln leicht wackelt kann er geformt werden und eine Stunde im Kühlschrank warten.
In der Zeit den Ofen vorheizen und backen werden wie das oben beschriebene ultimative Sauerteigbrot.
}

\rezept{Sauerteigstarter}{20}{etwa 170 kcal pro 100g}{
  & & (Roggen-) Vollkornmehl \\
  & & Weizenmehl \\
  & & lauwarmes Wasser (25 C) \\
}{
In einem Halbliterglas mit aufgelegtem Deckel an einem dunklen, etwa 25 Grad warmen Ort lagern.
Das Gewicht des Glases vorher notieren und Mengen mit einer Waage ausmessen.

Tag 1:
150g handwarmes Wasser und
100g Bio-Roggenvollkornmehl mischen bis die Masse keine Klumpen mehr hat. Mit einem lose sitzenden Deckel abdecken und 12 bis 24 Stunden ruhen lassen.

Tag 2+3:
70g Starter,
50g Bio-Roggenvollkornmehl,
50g ungebleichtes Weizenmehl und
115g handwarmes Wasser

Tag 4+5:
70g Starter,
50g Bio-Roggenvollkornmehl,
50g ungebleichtes Weizenmehl und
100g handwarmes Wasser

Tag 6:
50g Starter,
50g Bio-Roggenvollkornmehl,
50g ungebleichtes Weizenmehl und
100g handwarmes Wasser

Tag 7 und alle weiteren:
25g Starter,
50g Bio-Roggenvollkornmehl,
50g ungebleichtes Weizenmehl und
100g handwarmes Wasser

Im Schrank bei etwa 25 Grad lagern: jeden Tag füttern.

Kann im Kühlschrank gelagert werden, dafür 25g Starter mit 100g Mehl und 100ml Wasser füttern und zwei Stunden später in den Kühlschrank stellen.
Gefüttert wird alle 5 bis 7 Tage.

Kann eingefroren werden, dafür ein paar Stunden nach dem Füttern den aktiven Starter einfrieren.

Kann getrocknet werden, dafür auf einem Backpapier ausstreichen und trocknen, zerbröseln und in einem geschlossenen Gefäß lagern.

Tipp: Nicht mehr benötigten Starter mit etwas Wasser und optional einem Ei mischen und in einer geölten Pfanne zu einem Pfannkuchen backen, z.B. mit Sesam und Frühlingszwiebeln oder mit Rosinen und Zucker bestreuen.

Sauerteigstarter ist aktiv wenn er sein Volumen in etwa verdoppelt hat und gerade dabei ist herunterzusinken (das sollte etwa 5-6 Stunden nach dem Füttern der Fall sein, bei 25 Grad).}

