\rezept{
  Tapiokaperlen
}{20}{
  für Bubble tea
}{
3/4 & Tasse & Tapiocastärke \\
  2 & TL    & (taiwanesischer) brauner Zucker \\
  4 & TL    & heißes Wasser \\
}{
Das Wasser auf mittlerer Stufe erhitzen und den Zucker darin auflösen, dann die Hitze abstellen.
Darin 1 TL Stärke auflösen und den Herd wieder anstellen um die Masse anzudicken, danach wieder von der Hitze nehmen.
Mit dem Rest der Stärke zu einem klebrigen Teig vermengen und den Teig kneten bis er homogen und elastisch wird.
Den Teig in dünne Rollen formen und daraus kleine Perlen rollen, beim Kochen werden diese größer werden.

In Wasser 20 Minuten kochen, dann 20 Minuten stehen lassen, danach aus dem Wasser nehmen.

Traditionell werden sie als Bubbletea in einem schwarzen Tee mit Milch serviert.

Optional vor dem Servieren einen schnellen Sirup aus braunem Zucker machen und die Perlen damit bedecken.

Die ungekochten Teigperlen können in einem luftdichten Gefäß bis zu sechs Monate bei kühler Raumtemperatur aufbewahrt werden (im Kühlschrank werden sie ein wenig härter).
Sie können auch eingefroren werden und später zum Kochen direkt ins Wasser gegeben werden.
}
