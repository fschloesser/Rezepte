\rezept{
Auberginenbaellchen
}{20}{
vegane Fleischklopse
}{
1,5 & Tassen & Semmelbroesel \\
0,5 & Tasse & Olivenoel \\
3 & TL & puerierte Karotten oder \\
1 & & Ei \\
1 & & Aubergine \\
0,7 & Tassen & gekochte Linsen \\
2 & Zehen & kleingeschnittenen Knoblauch \\
0,25 & Tassen & geschnittenen Basilikum \\
0,25 & Tassen & geschnittene Petersilie \\
0,3 & Tassen & Kichererbsenmehl oder Hefeflocken \\
& & gemahlene Chiliflocken \\
& & Pfeffer und Salz \\
1 & Tasse & Pflanzenoel \\
}{
Den Ofen auf 150 Grad Celsius vorheizen.
In einer Schuessel die Semmelbroesel mit 2 TL Olivenoel und Salz vermischen und auf einem mit Backpapier ausgelegtem Blech im Ofen ca 10 Minuten roesten, dabei zwischendurch wenden.
Den Ofen auf 190 Grad hochschalten.
Das Blech mit Alufolie auslegen und die Aubergine darauf legen, vorher 3 bis 4 mal mit einer Gabel einstechen und dann im Ofen etwa 45 Minuten roesten bis diese komplett durchgegart ist.
Die Aubergine aus dem Ofen nehmen und (mit einer Schere) ein X in den Boden schneiden, in ein Sieb legen und ueber der Spuele etwa 20 Minuten austropfen lassen.
Nun die Aubergine laengs aufschneiden, das Fleisch herausloeffeln und mit dem (Ei oder) Karottenpueree, gekochten Linsen, Knoblauch, Basilikum, Petersilie und (den Hefeflocken oder) dem Kichererbsenmehl, Salz, Pfeffer und Chiliflocken zusammenpuerieren.
Die geroesteten Semmelbroesel hinzugeben und kurz untermixen.
Die Mischung mindestens 20 Minuten oder ueber Nacht luftdicht im Kuehlschrank ziehen lassen.
Die Masse in Golfballgrosse Baellchen formen und in einer Pfanne bei mittlerer Hitze in Olivenoel ca 5 Minuten frittieren.
Auf Kuechenpapier abtropfen lassen und direkt servieren.
}
