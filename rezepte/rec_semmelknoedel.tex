\rezept{Semmelknoedel}{20}{
etwa 4 Knödel}{
4   &    & altbackene Brötchen \\
2   &    & Eier \\
1   & EL & Butter \\
200 & ml & warme Milch \\
    &    & Salz \\
    &    & Pfeffer \\
    &    & Muskatnuss \\
1   &    & Zwiebel \\
    &    & Petersilie
}{
Brötchen in kleine Würfel schneiden, mit Salz, Pfeffer und Muskatnuss würzen.
Warme Milch über die Mischung gießen, alles vermengen und eine halbe Stunde ruhen lassen.

Petersilie und Zwiebel klein hacken.
Butter erhitzer, die Zwiebeln darin glasig dünsten.
Am Schluss Petersilie dazu geben und mit der Brötchenmischung vermengen.

Die Eier verquirlen und über die Brotwürfel geben.

Mit feuchten Händen vermischen und Knödel formen.
Diese in kochendem Wasser 15 bis 20 Minuten ziehen lassen.

Schmecken gut zum Beispiel zu Pilzsoße, siehe Seite \cite{rec:pilzsosse}. }
