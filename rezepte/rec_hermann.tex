\rezept{
  Hermann
}{20}{
  Kuchen
}{
200 &     g & Portion Hermann-Teig von Tag 10 //
200 &     g & Weizenmehl //
100 &     g & Zucker //
200 &    ml & Milch //
100 &    ml & Öl //
  3 &       & Eier //
  1 &  Pck. & Vanillezucker //
  1 & Prise & Salz //
  2 &    TL & Backpulver //
100 &     g & optional: geriebene oder gehackte Nüsse, Mandeln, Rosinen oder Schokostreusel //
}{
Alle Zutaten zu einem glatten Teig verrühren.
In eine gefettete, bemehlte Kastenform oder Gugelhupfform füllen.
Im vorgeheizten Backofen bei 180 °C (Umluft 160 °C) circa 45 bis 55 Minuten backen. Zum Ende der Backzeit eine Stäbchenprobe machen.
}

\rezept{
Pflegeanleitung für Hermann-Teig
}{20}{
100 &  g & Weizenmehl //
150 &  g & Zucker //
150 & ml & Milch //
}{
Egal ob geschenkt oder nach oben stehender Anleitung selbst hergestellt – von nun an wird der Hermann-Teig im Kühlschrank aufbewahrt.

Damit sich der Hermann-Teig vermehrt und schließlich zum Backen verwendet werden kann, wird er so gepflegt:

\begin{itemize}
\item  1. Tag: Ruhen lassen.
\item  2. Tag: Umrühren.
\item  3. Tag: Umrühren.
\item  4. Tag: Umrühren.
\item  5. Tag: Füttern – 50 g Weizenmehl, 75 g Zucker und 75 ml Milch zugeben. Gut verrühren.
\item  6. Tag: Umrühren.
\item  7. Tag: Umrühren.
\item  8. Tag: Umrühren.
\item  9. Tag: Umrühren.
\item 10. Tag: Füttern – 50 g Weizenmehl, 75 g Zucker und 75 ml Milch zugeben. Gut verrühren.
\end{itemize}
}
