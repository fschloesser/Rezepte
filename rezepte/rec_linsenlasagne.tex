\rezept{Linsenlasagne}{20}{vegetarisch}{
12 & & helle Lasagnebl\"atter \\
50 & g & Parmesan \\
800 & g & Bechamelso\ss{}e \\
& & \\
175 & g & getrocknete Linsen \\
& & oder \\
450 & g & Linsen aus der Dose \\
& & \\
750 & g & Tomaten \\
350 & g & M\"ohren \\
200 & g & Stangen Staudensellerie \\
100 & g & Porree \\
100 & g & Zwiebel \\
1 & & Knoblauchzehe \\
1 & & Chilischote \\
 & & \\
3 & EL & Oliven\"ol \\
2 & EL & Tomatenmark \\
2 & EL & Rotwein \\
200 & ml & Gem\"usebr\"uhe \\
& & Thymian und Rosmarin \\
& & Pfeffer und Salz \\
& & Cayennepfeffer
}{
\textbf{F\"ur die Bolognese:}

Die Linsen eventuell einweichen und in reichlich Wasser gar kochen.
(Dauert mindestens 3 Stunden + 30 Minuten)
Dosenlinsen abtropfen lassen.

Die Tomaten kreuzweise einritzen, mit kochendem Wasser \"uberbr\"uhen, kalt absp\"ulen und h\"auten.
M\"ohren und Sellerie sch\"alen, Porree putzen.
Das Gem\"use in kleine W\"urfel schneiden.
Zwiebel und Knoblauch abziehen und klein hacken.
Chili absp\"ulen, entkernen, hacken.

Das \"Ol in einem Topf erhitzen und das Gem\"use darin kurz and\"unsten, das Tomatenmark und etwas Br\"uhe dazu geben.
Alles etwa 10-15 Minuten bei mittlerer Hitze kochen lassen.
Linsen abtropfen lassen und dazu geben.

Thymian und Rosmarin absp\"ulen, trocken tupfen, klein hacken und unter das Gem\"use heben.
Die Bolognese mit Rotwein, Salz, Pfeffer und Cayennepfeffer abschmecken.

\textbf{Für die Bechamelsoße:}

Siehe Seite \ref{rec:bechamel}.

\textbf{Schichtanleitung:}

Den Boden einer Auflaufform mit 3 EL Bechamelso\ss{}e bestreichen.
Die Form mit Lasagnebl\"attern auslegen.
Sollten sich die Bl\"atter \"uberlappen, etwas So\ss{}e dazwischengeben, damit die Doppelschicht nicht zu trocken wird.

Darauf eine etwa 1 cm dicke Schicht Bolognese gleichm\"a\ss{}ig verstreichen.
Einige Essl\"offel Bechamelso\ss{}e auf der Bolognese verteilen.
Danach wieder Lasagnebl\"atter darauflegen und so weiterschichten, bis alle Zutaten verbraucht sind.

Zum Schluss mit einer Schicht Lasagnebl\"atter und Bechamelso\ss{}e abschließen, dabei sollten die Lasagnebl\"atter gut mit So\ss{}e bedeckt sein, damit sie nicht austrocknen.

Mit etwa 50g Parmesan bestreuen und auf der unteren Schiene bei 160 Grad Umluft 40-60 Minuten backen.
Wird die Kruste zu dunkel, mit einem zweiten Blech die Hitze abschirmen.

\textbf{Ratschlag:}
Lasagne vorm Anschneiden etwa 20 Minuten ruhen lassen, dann ist sie leichter zu portionieren. }
