\rezept{Sauerkraut}{20}{
selbstgemacht
}{
1 & & Weißkohl \\
einige & & Karotten \\
& & Salz \\
}{
Gemüse in Streifen schneiden (zum Beispiel mit der Mandoline) und 10 g bis 20 g Salz pro Kilo Gemüse hinzugeben.

Mit den Händen vermischen und so lange kneten bis das Kohlwasser das Gemüse bedeckt.

Mit einem Kohlblatt bedecken und etwa zwei Wochen lang bei Zimmertemperatur stehen lassen.

Nach etwa zwei Wochen, wenn keine Blasen mehr aufsteigen, im Kühlschrank lagern, dann beginnt die zweite, langsamere Phase der Fermentation.
Ab jetzt kann das Kraut gegessen werden, wird aber mit der Zeit besser.  }
