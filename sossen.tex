%\rezept{Titel}{20}{Untertitel}{150 & ml & Ei \\3 & Laib & Brot }{Ei aufs Brot spiegeln und mit Seffer und Pfalz bestreuen.}

\rezept{Bechamelso\ss{}e}{20}{reicht knapp f\"ur eine Linsenlasagne}{
50 & g & Butter \\
50 & g & Mehl \\
600 & g & Gem\"usebr\"uhe \\
100 & g & Schlagsahne \\
50 & g & frisch geriebener Parmesan \\
 & & frisch geriebene Muskatnuss
}{
Die Butter in einem Topf zerlassen und das Mehl darin unter R\"uhren mit einem Holzl\"offel and\"unsten, bis es Blasen wirft und Butter und Mehl sich gut miteinander verbunden haben.
Daf\"ur zun\"achst einen Holzl\"offel nehmen, damit die Mehl-Butter-Mischung nicht im Schneebesen h\"angen bleibt.

Br\"uhe (Zimmertemperatur) nach und nach dazugießen und dabei kr\"aftig mit einem Holzl\"offel r\"uhren, damit keine Kl\"umpchen entstehen.
Unter st\"andigem R\"uhren bei kleiner Hitze etwa 3 Minuten kochen.
Daf\"ur nun den Schneebesen nehmen, damit die So\ss{}e garantiert glatt und sch\"on cremig wird.

Hitze ausschalten, Sahne und die H\"alfte des geriebenen Parmesans unterr\"uhren und die So\ss{}e mit Salz, Pfeffer und frisch geriebenem Muskat abschmecken (Muskat sparsam dosieren!).
Topf mit der So\ss{}e vom Herd nehmen.  }

\rezept{Vegane Mayonnaise}{100}{gehaltvoll}{
150 & ml & "Ol \\
50 & ml & Sojamilch \\
1 & EL & Senf \\
1 & Prise & Salz \\
1 & Prise & Pfeffer \\
$1/2$ && Zitrone
}{"Ol, Sojamilch, Senf und Zitronensaft mit einem Stabmixer
zu einer cremigen Masse p"urieren. Mit Salz und Pfeffer abschmecken.

Tip: "Ol und Sojamilch sollten etwa die gleiche Temperatur haben.}

\rezept{Pistazienpesto}{20}{
zum Beispiel zum Grillkartoffelsalat}{
1 & Zehe & Knoblauch \\
3 & EL & Basilikum \\
4 & EL & Pistazienkerne \\
& & ungesalzen \\
5 & EL & Mayonnaise \\
2 & TL & Wei\ss{}weinessig \\
1/2 & TL & grobes Meersalz \\
1/4 & TL & Pfeffer
}{
Den Knoblauch zerkleinern.
Basilikum und Pistazien hinzuf\"ugen und alles fein hacken.
Die Mischung in eine gro\ss{}e Sch\"ussel geben und mit den \"ubrigen Zutaten vermischen.
}

