\documentclass[9pt, a4paper, twoside, twocolumn]{book} % set draft / final
% packages
% -tex
\usepackage[T1]{fontenc}
\usepackage[utf8]{inputenc}
\usepackage[ngerman]{babel}
\usepackage{titlesec} % A package providing an inter­face to sectioning commands for selection from various title styles
\usepackage[margin=2cm]{geometry}
\geometry{bindingoffset=0cm}
% -graphics/floats
\usepackage{graphicx}
\usepackage{xcolor}
\usepackage{wrapfig}
\usepackage{framed}
% -lists and tables
\usepackage{paralist}
\usepackage{tabularx}
% -index
\usepackage{makeidx}
\makeindex
% -text
\usepackage{ragged2e}
%\usepackage[activate, tracking=true, kerning=true]{microtype}
%	\SetTracking{}{500}
%\usepackage{libertine}

% make table of contents clickable
\usepackage{hyperref}
\hypersetup{
    colorlinks,
    citecolor=black,
    filecolor=black,
    linkcolor=black,
    urlcolor=black
}
% graphicspath
\graphicspath{{img/}}

\makeatletter
\newcommand\HUGE{\@setfontsize\Huge{40}{50}}
\newcommand\HUUGE{\@setfontsize\Huge{90}{110}}
\newcommand\GRANDE{\@setfontsize\Huge{120}{0}}
\newcommand\GRAANDE{\@setfontsize\Huge{250}{0}}
%\def\WFfill{\par
%    \ifx\parshape\WF@fudgeparshape
%    \nobreak
%    \ifnum\c@WF@wrappedlines>\@ne
%    \advance\c@WF@wrappedlines\m@ne
%    \vskip\c@WF@wrappedlines\baselineskip
%    \global\c@WF@wrappedlines\z@
%    \fi
%    \allowbreak
%    \WF@finale
%    \fi
%}
\makeatother

% chapter heading (provided by package titlesec)
%\titleformat{command}[shape]{format}{label}{sep}{before-code}[after-code]
\titleformat{\chapter}[block]{\GRAANDE}{}{0em}{\GRANDE\thechapter\quad}[]
\titleformat{\section}[block]{\huge \scshape \centering}{\thesection}{0.5em}{}[\vspace{-2.5ex}%
\rule{\linewidth}{0.3pt}]

%\newpagestyle{mystyle}{
%  \sethead{}{}{}
%  \setfoot{}{\thepage}{}}
%\pagestyle{mystyle}

%%% Tips %%%
%\tips{Titel}{Content}
\newcommand{\tips}[2]{
\vspace{30pt}
\hspace{15pt}\hrulefill {\huge \textsc{#1} } \hrulefill
\vspace{20pt}

\begin{compactitem}
#2
\end{compactitem}
}

%%% Rezept %%%
%\rezept{Titel}{stretch}{Untertitel}{Zutaten als Tabelle}{Anleitung}
\newcommand{\rezept}[5]{
\section*{\textsc{#1}}
\index{#1}

\begin{flushright}
\textit{#3}
\end{flushright}

\textbf{Zutaten:}

\begin{tabularx}{\linewidth}{rl X}
#4
\end{tabularx}

\textbf{Anleitung:}

#5
}

%%%%%%%%%%%%%%%%
%%% DOCUMENT %%%
%%%%%%%%%%%%%%%%

\begin{document}
%% NEW COMMANDS %%
\renewcommand{\indexname}{Stichwortverzeichnis}

%%% FORMAT %%%
\pagenumbering{gobble}
\setlength\parindent{0pt}

%%% TITLEPAGE %%%
\thispagestyle{empty}
\vspace*{3cm}
\begin{center}
  Rezepte
\vspace{2cm}
\end{center}
\vspace{3cm}
\newpage

\tableofcontents

\cleardoublepage
\printindex

%%% CONTENT %%%
\cleardoublepage
\pagenumbering{arabic}

\chapter{K\-ü\-c\-h\-e\-n\-t\-i\-p\-s}
% kuechentips
\thispagestyle{empty}
%\newpage
Küche
\begin{compactitem}
\item Messer immer scharf halten, nicht über das Brettchen ziehen um geschnittenes Gemüse herunter zu schieben.
\item Ein Handtuch in Griffweite halten.
\item Nicht Reiben (microplane zester), Mandolinen, Juliennes vergessen (dafür gibt es auch Schutzhandschuhe)
\item Siebe und Chinoissiebe benutzen um Soßen zu passieren.
\item Utensilien im Restauranthandel, nicht im Küchengeschäft kaufen.
\item Restaurantgeheimnisse sind: Butter, Schalotten, Brühe, Salz und Säure
\item Vor dem Kochen Zutaten waschen, schneiden, vorbereiten und abwiegen (mise-en-place).
\item Ein Brettchen oder eine Schale auf ein nasses Handtuch stellen um es an Ort und Stelle zu halten.
\item Eine Kompost-Schüssel ausgelegt mit einer alten Zeitung benutzen anstatt einer Mülltüte.
\item Die Küche beim Arbeiten sauber machen und sauber halten, immer genug Arbeitsplatz sauber und frei haben.
\end{compactitem}

Lagerung
\begin{compactitem}
\item Soße in Eiswürfelformen einfrieren.
\item Tomatensoße zu sauer? Länger kochen oder Zucker hinzufügen.
\item Avocados in einer Papiertüte mit einer Banane reifen schneller, im Kühlschrank reifen sie langsamer.
\item Fleisch oder Lasagne: Nach dem Kochen vor dem Anschneiden ruhen lassen.
\item Gemüse und frische Kräuter in einer Dose frisch halten mit einem Handtuch zusammen, um überschüssige Feuchtigkeit aufzunehmen.
\item Ausgekratzte Vanilleschoten in einem Glas mit Zucker aufbewahren um Vanillezucker zu machen.
\item Gemüsereste, Knoblauch- und Zwiebelschalen, Knochen aufbewahren und einfrieren um eigene Brühe zu machen.
\end{compactitem}

Kochen
\begin{compactitem}
\item Die Pfanne nicht zu voll machen.
\item Säure hinzufügen gegen Salz (zum Beispiel Essig, Zitronensaft oder Wein)
\item Soßen andicken mit Stärke: 1 TL Stärke mit 1 TL Wasser vermischen und hinzufügen.
\item Frische Gewürze in Butter oder Öl einfrieren.
\item Soße oder Kompott zu sauer? Etwas Natron (1/4 TL) hinzufügen, aber nicht zu viel!
\item Gewürze oder Nüsse anrösten um den Geschmack zu verbessern.
\item Auf kleiner Hitze kochen: Gemüse fällt nicht so schnell auseinander und Fleisch wird nicht trocken und zäh.
\item Großzügig sein mit Salz.
\item Immer etwas Säure zum Essen hinzufügen.
\item Pfannenreste ablöschen und als Soßenbasis benutzen.
\item Nudeln in der Soße zu Ende kochen. Falls die Soße zu dick wird, mit etwas aufbewahrtem Nudelwasser verlängern.
\item Geriebenen Käse oder Käsewürfel einfrieren und zum überbacken benutzen.
\item Gemüse frittieren um es zu karamellisieren und den Geschmack zu verbessern.
\item Beschichtete Pfannen nur für Eier, Pfannkuchen und French-toast benutzen. (Beschichtung hält Hitze nicht gut aus)
\end{compactitem}

Backen
\begin{compactitem}
\item Äpfel zum Backen: Jonagold, Elstar, Boskoop, Braeburn (hoher Säuregehalt, fest, nicht mehlig)
\item Einfetten nur mit Butter oder Margarine. Öl hinterlässt schwer zu reinigenden Film (verharzt).
\item Gefrorene Butter mit einer Reibe benutzen.
\end{compactitem}

Teige
\begin{compactitem}
\item Lange geknetete Teile kleben, weil der Kleber (das Gluten) im Mehl aktiviert wird.
\item Keksteig vor dem Backen in den Kühlschrank stellen und im vorgeheizten Ofen backen, dann zerlaufen sie nicht.
\item Brotteig lang und gut kneten um eine gute Struktur zu erhalten. Keks- und Kuchenteig so wenig wie möglich kneten.
\end{compactitem}

Fermentation
\begin{compactitem}
\item Beste Temperatur: 20 Grad. Im Kühlschrank verlangsamt die Fermentation und die Haltbarkeit verlängert sich auf mehrerer Monate.
\item Wasser und Seife genügen, es braucht nicht alles steril zu sein.
\item Während der Fermentation kann eine weiße Hefeschicht auf der Oberfläche entstehen, das ist normal und nicht schädlich und kann einfach entfernt werden.
\item Die Gefäße nicht luftdicht abschließen, Luft muss entweichen können.
\item Luftdicht abgeschlossene Fermentationen entwickeln eine Art Sprudel.
\item Das Fermentat muss komplett unter Wasser liegen.
\item Die Salzlake sollte etwa 2 Prozent betragen.
\item Fermentationszeit ist zwischen 2 Tagen und 2 Wochen.
\item Süße Dinge fermentieren schneller.
\end{compactitem}

\chapter{S\-a\-l\-a\-t\-e}
% salate
\thispagestyle{empty}
%\newpage
\rezept{Bohnensalat}{20}{
a la Mama}{
1 & Becher & saure Sahne \\
1 & & Zwiebel \\
1 & Zehe & Knoblauch \\
& & gr\"une Bohnen
}{
Für die Soße die saure Sahne mit Salz, Pfeffer, Zucker, Essig und geriebenen Knoblauch vermischen.

Bohnen mit Bohnenkraut und Zwiebeln kochen und warm in die Soße geben. }

\rezept{Brotsalat - Panzanella}{20}{Verwertung f\"r altes, hartgewordenes Brot}{
 & & Feldsalat \\
 & oder & Rucola \\
 & etwas & altes Brot \\
1 & & Paprika \\
1/2 & & Gurke \\
 & & (Cocktail)tomaten \\
 & & (Walnuss)\"ol
}{
Gem\"use und Brot in kleine W\"urfel schneiden, Salat putzen.
Alles in eine Sch\"ussel geben, mit \"Ol betr\"aufeln, salzen und pfeffern.

Ziehen lassen - fertig.  }

\rezept{Coleslaw}{20}{
aka Krautsalat}{
1,5 & kg & Weißkohl \\
450 & g  & Möhren \\
1   &    & Zitronen \\
2   & TL & Salz \\
2   & TL & Zucker \\
50  & ml & Milch \\
30  & ml & Öl \\
50  & g  & Joghurt \\
1   & TL & Essig
}{
Weißkohl und Möhren in feine Streifen schneiden, zum Beispiel mit einer Mandoline und in einer Schüssel mit Salz und Zucker vermischen.

Das Kraut kneten bis die gewünschte Konsistenz erreicht ist und sich am Schüsselboden Wasser sammelt, 20 min ziehen lassen.

Milch mit einem Pürierstab mixen und das Öl in kleinen Portionen nach und nach hinzugeben, Joghurt und Essig dazugeben, pfeffern.

Zitronen auspressen und mit der Soße über den Salat geben, nochmal ziehen lassen.}

\rezept{Kartoffelsalat}{20}{mit Grillkartoffeln und Grillgem\"use, 4-6 Personen, 20 min}{
& & Pistazienpesto \\
1 & kg & mittelgro\ss{}e \\
& & festkochende Kartoffeln \\
& & grobes Meersalz \\
2 & & grosse Paprika \\
2 & EL & Oliven\"ol \\
& & Pfeffer \\
2 & EL & frisches Basilikum
}{
Die Kartoffeln waschen, achteln, in einen gro\ss{}en Topf geben und gut mit Wasser bedecken.
2 TL Salz hinzuf\"ugen und aufkochen.
Die Hitze reduzieren und die Kartoffeln 5-10 Min kochen, bis sie knapp gar sind.
In der Zwischenzeit die Paprikaschoten l\"angs halbieren und in 3 cm breite St\"ucke schneiden.

Die Kartoffeln abgie\ss{}en und zur\"uck in den leeren Topf geben.
Die Paprikast\"ucke sowie 2 EL \"Ol und 1/2 TL Salz hinzuf\"ugen und alles gut vermischen.

Den Grill f\"ur mittlere Hitze vorheizen.
Sobald sie hei\ss{} ist, Kartoffeln und Paprika m\"oglichst flach auf einem mit Backpapier ausgelegten Blech verteilen.
Etwa 10 - 15 Min grillen bis die Kartoffeln auf allen Seiten knusprig braun und innen weich sind.
Gelegentlich wenden.
Das Gem\"use in eine Sch\"ussel mit dem Pistazienpesto geben.
Vorsichtig mischen bis das Gem\"use \"uberall mit Pesto bedeckt ist.
Mindestens 5 Minuten abk\"ulen lassen und mit Salz und Pfeffer abschmecken.

Warm oder auf Zimmertemperatur abgek\"ult servieren. }

\rezept{
  Instantnudelsalat
}{20}{
  aus Nepal
}{
 & & Instant Nudeln \\
 & & Tomate \\
 & & Zwiebel \\
 & & Ingwer \\
 & & Chili \\
}{
Die Nudeln in der Tüte zerkleinern und in eine Schüssel geben.
Das Gemüse klein schneiden und mit den Nudeln vermengen.
Etwas Limettensaft und die Nudelwürzmischung hinzugeben.
Vermischen und kurz ziehen lassen.
}

\input{rezepte/rec_linsensalat}
\rezept{Obstsalat}{20}{erfrischend}{
& & Obst \\
& & Honig \\
& & Orange \\
& & Limette \\
frische & & Minze
}{
Für das Dressing die Orange und Limette auspressen und mit Honig abschmecken, wer mag kann auch noch Zesten dazugeben.

Das Obst und die Minze klein schneiden und mit dem Dressing bedecken, ziehen lassen. }

\chapter{D\-r\-e\-s\-s\-i\-n\-g\-s}
% dressings
\thispagestyle{empty}
%\newpage
\rezept{
Chimichurri
}{20}{
}{
1   & Tasse & gehackte Petersilie \\
1   & Bund  & gehackte Frühlingszwiebeln \\
2   & EL    & getrockneter Oregano \\
4   & Zehen & Knoblauch \\
1   &       & Zitrone \\
2   & EL    & (Rotwein)essig \\
1/2 & Tasse & Oliven\"ol \\
1   &       & Spitzpaprika oder Chili \\
    &       & Salz \\
    &       & Pfeffer \\
}{
Zutaten schneiden/hacken und dann alles pürieren und mindestens einen Tag ziehen lassen. }

\rezept{Honig-Senf-Vinegraitte}{20}{
Klassiker}{
1 & EL & Honig \\
1 & EL & Senf \\
2-3 & EL & Balsamicoessig \\
10 & EL & Oliven\"ol \\
 & & Sonnenblumen\"ol \\
 & & Salz \\
 & & Pfeffer \\
}{
Erst Salz, Pfeffer, Essig, Honig und Senf glatt r\"uhren, dann mit dem Oliven\"ol verr\"uhren.
Zum Schluss mit dem Mixstab p\"urieren und das geschmacksneutrale \"Ol hinzugeben bis die gew\"unschte Konsistenz erreicht ist.
Tipp: Man kann daraus auch ein Joghurtdressing machen.
Dazu tausche man den Balsamicoessig und die \"Ole durch einen Becher Joghurt aus.
}

%\newpage
\rezept{Kartoffelsalat-Dressing}{20}{
ohne Mayo, lecker}{
1 & Becher & Creme Fraiche oder Joghurt (10\%) \\
 & etwas & Senf \\
 & etwas & \"Ol \\
 & etwas & Honig }{
Alles vermengen, mit Salz und Pfeffer abschmecken, fertig. }

\rezept{Pebre}{20}{pica - scharf}{
1 & & große Tomate \\
1 & & mittelgroße Zwiebel \\
2 & Zehen & Knoblauch \\
3 & EL & gehackte Petersilie \\
3 & EL & gehackter Koriander \\
1 & & grüne Chilipaprika \\
1 & EL & rote Chilipaste \\
2 & EL & (Oliven)öl \\
& & Salz und Pfeffer \\
1 & EL & Weißwein(essig) \\
& etwas & Zitronensaft
}{
Alles schneiden und vermischen, mindestens eine Stunde ziehen lassen und nachwürzen, mit Brot und Butter servieren.}

\rezept{Zitronendessing}{20}{
f\"ur S\"u\ss{}es}{
1 & Becher & saure Sahne \\
1 & & Zitrone \\
& & Zucker
}{
Alles zusammen kippen und nach Geschmack s\"u\ss{}en.
}

\chapter{S\-u\-p\-p\-e\-n}
% suppen
\thispagestyle{empty}
%\newpage
\rezept{Gazpacho 1}{20}{F\"ur hei\ss{}e Tage und vier Portionen}{
1 & kg & Tomaten \\
4 & Zehen & Knoblauch \\
1 & & Gem\"usezwiebel \\
2 & kleine & Paprikaschoten \\
6 & EL & Oliven\"ol \\
4 & Scheiben & Toastbrot \\
1 & kl. Dose & Tomaten \\
1/2 & l & Br\"uhe oder Wasser \\
 & & Salz und Pfeffer
}{
Das Gem\"use putzen und in St\"ucke schneiden (die Tomaten brauchen nicht gesch\"alt zu werden.
Alle Zutaten werden im Mixer p\"uriert, am Besten in mehreren Partien, wobei jedesmal etwas Br\"uhe gegeben werden muss.
Auch das Toastbrot wird mit p\"uriert, es dient der Bindung.
Am Schluss l\"asst man das \"Ol mit in den Mixer flie\ss{}en.

In einer gro\ss{}en Sch\"ussel alles gut verr\"uhren und f\"ur mindestens eine Stunde im K\"uhlschrank gut durchk\"uhlen lassen.

Besonders gut mit frischem Baguette an hei\ss{}en Tagen.

Tip: Wer mag, kann in kleine W\"urfel geschnittene Tomate, Gurke und Zwiebel separat dazu reichen.
Die Gazpacho kann auch gut eingefroren werden.}

\rezept{Gazpacho 2}{20}{Andaluz, original}{
1 & kg & Kirschtomaten \\
1 & Schote & grüne Spitzpaprika \\
1 & & Gurke \\
2 & Zehen & Knoblauch \\
50 & ml & Olivenöl \\
50 & g & altes, hartes Brot \\
250 & ml & Wasser \\
5 & g & Salz \\
30 & ml & Essig
}{
Das Gem\"use putzen und in St\"ucke schneiden (die Tomaten brauchen nicht gesch\"alt zu werden.
Alle Zutaten werden mit dem Olivenöl, dem Wasser und dem Essig im Mixer p\"uriert.

Nach dem Pürieren wird die Suppe durch ein Sieb gegeben um Schalen und Kerne herauszufiltern.

In einer gro\ss{}en Sch\"ussel alles gut verr\"uhren und f\"ur mindestens eine Stunde im K\"uhlschrank gut durchk\"uhlen lassen.

Tip: Wer mag, kann in kleine W\"urfel geschnittene Tomate, Gurke und Paprika separat dazu reichen.}

%\newpage
\input{rezepte/rec_gemuesesuppe}
\input{rezepte/rec_kakaolinsensuppe}
\input{rezepte/rec_linsensuppe}
\rezept{Kürbis-Linseneintopf}{20}{simpel, 4 Portionen}{
1/2 & & Kürbis \\
300 & g & Linsen (rot, gelb) \\
& & Curry \\
& & Koriander \\
& & Ingwer \\
& & Rosinen \\
& optional & Petersilie
}{
Die Linsen waschen, den Kürbis in Würfel schneiden und beides im Topf kurz anbraten und mit Wasser ablöschen, sodass das Gemüse fast bedekt ist.
Salz, Pfeffer und Curry (nicht Goldelefant) hinzugeben (nach Geschmack Ingwer) und 10-15 Minuten weichkochen.

Mit einem Kartoffelstampfer den Kürbis ein wenig zerdrücken aber keinen Brei machen.
Nach Gusto mit Koriander und oder Rosinen servieren. }

\rezept{
Zwiebelsuppe
}{20}{
klassisch
}{
  1 & kg & Gemüsezwiebeln \\
  1 & dicke & Knoblauchzehe \\
  2 & EL & Tomatenmark \\
1,5 &  L & Brühe nach Eurer Wahl \\
350 & ml & Weißwein \\
  1 & EL & Honig \\
  2 & EL & mildes Paparikapulver \\
  1 & EL & Fenchelsamen \\
  1 & EL & Thymian \\
  1 & EL & Oregano \\
 & & Salz und Pfeffer \\
 & & Olivenöl
}{
Die Zwiebeln und den Knoblauch in Ringe schneiden und bei hoher Hitze in zwei EL Olivenöl anschwitzen.
Tomatenmark und Paprikapulver hinzugeben und kurz anschwitzen, mit dem Wein und der Brüe ablöschen.
Die restlichen Gewürze hinzugeben.

Wenn die Zwiebeln verkocht sind die Suppe in Schälchen geben und mit Croutons und Käse bestreuen.
Im Ofen übergrillen und servieren.
}

\chapter{H\-a\-u\-p\-t\-g\-e\-r\-i\-c\-h\-t\-e}
% hauptgerichte
\thispagestyle{empty}
%\newpage
\rezept{Champignon-Tomatenrahmso\ss{}e}{20}{
mit Basilikum}{
250 & g & Champignons \\
1 & & Zwiebel \\
2 & Zehen & Knoblauch \\
3 & & Tomaten \\
3 & EL & Tomatenmark \\
& etwas & Basilikum \\
150 & ml & Sahne \\
1 & TL & Gem\"usebr\"uhe \\
300 & g & Nudeln \\
& & Petersilie
}{
Die Nudeln normal kochen.

F\"ur die So\ss{}e die Zwiebeln und den Knoblauch klein schneiden, die Champignons putzen und in d\"unne Scheiben schneiden, die Tomaten klein w\"urfeln.

Zuerst die Zwieblen mit dem Knoblauch in einer Pfanne mit etwas \"Ol leicht anbraten, die Champignonscheiben dazugeben und mit anbraten.
Die Tomaten sowie das Tomatenmark zuf\"ugen, mit der Sahne aufgie\ss{}en und bei niedriger Temperatur k\"ocheln lassen.

Falls die So\ss{}e zu dickfl\"ussig ist Milch hinzugeben.

Nun das Basilikum hinzuf\"ugen und mit Pfeffer und Br\"uhe nach Geschmack w\"urzen.  }

\rezept{Els\"asser Flammkuchen}{20}{
Einfach und gut - f\"ur 4 Portionen}{
440 & g & Mehl (550) \\
200 & ml & lauwarmes Wasser \\
6 & EL & \"Ol \\
2 & & Eigelb \\
1  & TL & Salz \\
500 & g & Zwiebeln \\
1 & Becher & saure Sahne \\
1 & Becher & s\"u\ss{}e Sahne
}{
Das Mehl in eine Sch\"ussel sieben, Wasser, \"Ol, Eigelb und Salz zugeben, zu einem glatten Teig verkneten und zugedeckt 30 Minuten an einem zimmerwarmen Ort ruhen lassen.

Zwiebeln halbieren, in feine Scheiben hobeln.
S\"u\ss{}e und saure Sahne mit etwas Salz und Pfeffer verquirlen.

Teig vierteln.
Jedes St\"uck auf einem gro\ss{}en St\"uck Backpapier zu einer hauchd\"unnen rechteckigen Platte auswellen.
Sahnemischung darauf verstreichen und mit Zwiebeln bestreuen.

Bei Umluft 220 Grad etwa 10-15 Minuten goldbraun backen.  }

%\newpage
\rezept{Gem\"usecurry}{20}{
Vegan, 90 min}{
2 & & Zwiebeln \\
3 & & Knoblauchzehen \\
150 & g & gesch\"alte \\
 & & Dosentomaten \\
2 & EL & Oliven\"ol \\
1 & EL & gelbe Currypaste \\
200 & g & Linsen \\
1-2 & TL & Kurkuma \\
400 & ml & Gem\"usebr\"uhe \\
200 & g & Blumenkohl \\
2 & & M\"ohren \\
3 & St & Staudensellerie \\
150 & g & Erbsen \\
150 & ml & Kokosmilch \\
80 & g & Margarine \\
1 & Prs & Salz \\
& & frischer Koriander
}{ Zwiebeln und Knoblauch abziehen und klein w\"urfeln.
Die gesch\"alten Tomaten hacken.
Oliven\"ol in breitem Topf erhitzen, Zwiebeln und Knoblauch darin 2-3 Minuten anschwitzen.
Die Currypaste unterr\"uhren, Linsen, Kurkuma und gehackte Tomaten dazugeben.
Das Gem\"use mit der Br\"uhe abl\"oschen und ca. 1 Stunde zugedeckt bei schwacher Hitze leise k\"ocheln lassen.

Inzwischen den Blumenkohl in kleine R\"oschen teilen, M\"ohren und Staudensellerie sch\"alen und in schr\"age Scheiben schneiden.
Das vorbereitete Gem\"use und die Erbsen ca. 10 Minuten vor Ende der Garzeit zu den Linsen geben, dann Kokosmilch und Margarine hinzuf\"ugen.

Das Linsencurry mit Salz abschmecken und mit Koriander garniert servieren.
Dazu passt ged\"ampfter Basmatireis.  }

\rezept{Quark zu Kartoffeln}{20}{
a la Mama}{
& & (fetter) Quark \\
& & Tomaten \\
& & Paprika \\
& & Gurke \\
& & Schnittlauch \\
& & Zucker}
{Gem\"use sehr klein w\"urfeln und mit den restlichen Zutaten vermischen.

Nach Geschmack mit Salz, Pfeffer und Knoblauch würzen.

Tip: Auch lecker nur mit Tomaten. }

%\newpage
\input{rezepte/rec_koshary}
%\newpage
\rezept{Linsenlasagne}{20}{vegetarisch}{
& & helle Lasagnebl\"atter \\
20 & g & Butter \\
50 & g & Parmesan \\
& & Bechamelso\ss{}e \\
& & \\
175 & g & getrocknete Linsen \\
& & oder \\
450 & g & Linsen aus der Dose \\
5 & & Tomaten \\
2 & & M\"ohren \\
2 & & Stangen Staudensellerie \\
100 & g & Porree \\
1 & & Zwiebel \\
1 & & Knoblauchzehe \\
1 & & Chilischote \\
3 & EL & Oliven\"ol \\
200 & ml & Gem\"usebr\"uhe \\
2 & EL & Tomatenmark \\
2 & EL & Rotwein \\
& & Thymian und Rosmarin \\
& & Pfeffer und Salz \\
& & Cayennepfeffer
}{
F\"ur die Bolognese:

Die Linsen eventuell einweichen und in reichlich Wasser gar kochen.
(Dauert mindestens 3 Stunden + 30 Minuten)
Dosenlinsen abtropfen lassen.

Die Tomaten kreuzweise einritzen, mit kochendem Wasser \"uberbr\"uhen, kalt absp\"ulen und h\"auten.
M\"ohren und Sellerie sch\"alen, Porree putzen.
Das Gem\"use in kleine W\"urfel schneiden.
Zwiebel und Knoblauch abziehen und klein hacken.
Chili absp\"ulen, entkernen, hacken.

Das \"Ol in einem Topf erhitzen und das Gem\"use darin kurz and\"unsten, etwas Br\"uhe dazugie\ss{}en.
Alles etwa 10-15 Minuten bei mittlerer Hitze kochen lassen.
Linsen abtropfen lassen und dazu geben.

Thymian und Rosmarin absp\"ulen, trocken tupfen, klein hacken und unter das Gem\"use heben.
Die Bolognese mit Rotwein, Salz, Pfeffer und Cayennepfeffer abschmecken.

Schichtanleitung:

Den Boden einer Auflaufform mit 3 EL Bechamelso\ss{}e bestreichen.
Die Form mit Lasagnebl\"attern auslegen.
Sollten sich die Bl\"atter \"uberlappen, etwas So\ss{}e dazwischengeben, damit die Doppelschicht nicht zu trocken wird.

Darauf eine etwa 1 cm dicke Schicht Bolognese gleichm\"a\ss{}ig verstreichen.
Einige Essl\"offel Bechamelso\ss{}e auf der Bolognese verteilen.
Danach wieder Lasagnebl\"atter darauflegen und so weiterschichten, bis alle Zutaten verbraucht sind.

Zum Schluss mit einer Schicht Lasagnebl\"atter und Bechamelso\ss{}e abschließen, dabei sollten die Lasagnebl\"atter gut mit So\ss{}e bedeckt sein, damit sie nicht austrocknen.

Mit etwa 50g Parmesan bestreuen und die Butter in Fl\"ockchen daraufsetzen.
Die Auflaufform auf der unteren Schiene bei 160 Grad Umluft 40-60 Minuten backen.
Wird die Kruste zu dunkel, mit einem St\"uck Backpapier abdecken.

Tip: Lasagne vorm Anschneiden etwa 10 Minuten ruhen lassen, dann ist sie leichter zu portionieren. }

%\newpage
\rezept{M\"urbeteig}{20}{
für 16 Muffins, 6 kleine Tartes oder eine große}{
200 & g & Mehl \\
100 & g & Butter/Margarine \\
1/2 & TL & Salz \\
3 & EL & Wasser \\
1 & optionales & Ei
}{Teig kneten und in eine gefettete Backform geben.
Das Ei ist optional, ohne ist aber unerprobt.

Vorsicht: die Butter nicht weich werden lassen und den Teig nicht zu lang kneten.
Er ist fertig und muss in den Kühlschrank wenn er eine Masse ist aber man Butter und Mehl noch unterscheiden kann.

Tip: Für Muffinförmchen: für Boden Kreise ausstechen und separat auslegen, für den Rand einen Streifen schneiden und ankleben.

Mal probieren: süßer Mürbeteig (1,2,3-Teig), 1 Teil Zucker, 2 Teile Butter, 3 Teile Mehl.  }

\rezept{
Ofengemüse
}{20}{
einfach, schnell und lecker
}{
 & & Kürbis \\
 & & Süßkartoffel \\
 & & Möhren \\
 & & Kartoffeln \\
 & & Salz \\
 & & Olivenöl
}{
Das Gemüse in etwa 2cm große Stücke schneiden und auf ein Backblech legen.
Mit Salz bestreuen und mit Olivenöl besprenkeln.
Bei 200 Grad mit Ober- und Unterhitze 35 Minuten backen für saftiges Gemüse.
}

\rezept{Quiche}{20}{
für 16 Muffins, 6 kleine Tartes oder eine große}{
2-3 & & Eier \\
2 & & Tomaten \\
1 & & Zwiebel \\
6 & Zehen & Knoblauch \\
150 & ml & Milch \\
100 & g & Schafskäse \\
1 & & Zucchini \\
6 & & Karotten
}{
Mürbeteig mit Ei vorbereiten und kalt stellen wie oben beschrieben.
Gemüse reiben und mit Zwiebeln, Knoblauch in Olivenöl andünsten, Form fetten.
Eier und Milch mit Salz, Pfeffer und Kräutern vermischen.
Teig dünn ausrollen und in die Form drücken, mit dem Gemüse belegen und die Eiermischung darüber gießen.
Den Schafskäse darüber krümeln.
Bei 200 Grad etwa 30 bis 40 Minuten backen.
}

\rezept{
  Pizzateig
}{20}{
  original italienisch
}{
  1 &  l & Wasser (20 C) \\
1.5 & kg & Mehl \\
 50 &  g & feines Salz \\
  5 &  g & Trockenhefe \\
}{
Mehl und Trockenhefe mischen, Salz in Wasser auflösen, danach zur Mehlmischung geben und alles mixen.
15 Minuten kneten und 2 Stunden ruhen lassen.

Den Teig in 4 oder 6 kleinere Kugeln teilen und 4 weitere Stunden ruhen lassen.
}

\rezept{Pizza-\"Ol-Teig}{20}{
Einfach}{
500 & g & Mehl\\
1 & TL & Salz \\
6 & EL & \"Ol \\
1 & Pck. & Hefe \\
300 & ml & Wasser
}{
Mehl, Salz und \"Ol vermischen.
1 W\"urfel Hefe in die Mitte br\"oseln.
Wasser dazugeben und durchkneten bis der Teig sich vom Rand l\"ost.

Tuch dr\"uberlegen und ruhen lassen.  }


\rezept{Sauerkraut}{20}{
selbstgemacht
}{
1 & & Weißkohl \\
einige & & Karotten \\
& & Salz \\
}{
Gemüse in Streifen schneiden (zum Beispiel mit der Mandoline) und 10 g bis 20 g Salz pro Kilo Gemüse hinzugeben.

Mit den Händen vermischen und so lange kneten bis das Kohlwasser das Gemüse bedeckt.

Mit einem Kohlblatt bedecken und etwa zwei Wochen lang bei Zimmertemperatur stehen lassen.  }

\rezept{
  Spätzle
}{20}{
  traditional schwäbisch
}{
  100 & g & Mehl \\
  1 & & Ei \\
  & & Salz \\
  & etwas & Wasser \\
}{
  Zutaten zu einem zähflüssigen Teig mischen und in gesalzenes, kochendes Wasser schaben.
  Kochen bis die Spätzle an die Oberfläche schwimmen, heraus nehmen und servieren.
}

%\newpage
\rezept{Schmorgurken}{20}{auch mit Speck}
{1 & kg & Gurken\\
1/4 & l & Br\"uhe \\
40 & g & Speck \\
1 & & Zwiebel \\
1 & TL & Zucker \\
1 & EL & Mehl \\
 & & Dill\\
 & & Tomaten\\
 & & Kartoffeln als Beilage}{
Die Gurken sch\"alen, entkernen und in zweifingerbreite St\"ucke schneiden.
Den Speck und die Zwiebel w\"urfeln, dann anbraten.
Die Gurkenst\"uckchen hinzu und etwa 20 Minuten mitbraten, sodass sie glasig aussehen, aber noch ein wenig Biss haben.

Bei Bedarf kleingeschnittene Tomaten oder Kirschtomaten hinzu f\"ugen und kurz mitbraten.

Im Fett Zucker und Mehl br\"aunen, mit der Fleischbr\"uhe auff\"ullen, pikant mit Salz, Pfeffer nud Essig abschmecken.
\"Uber die geschmorten Gurken geben, kleingehackten Dill dazu, umr\"uhren und noch ein wenig durchziehen lassen.

Dazu Kartoffeln reichen.}

\input{rezepte/rec_semmelknoedel}
\rezept{Volognese}{20}{Vegane Bolognese}{
500 & g & Tofu \\
250(?) & ml & Rotwein \\
8 & EL & Tomatenmark \\
200(?) & g & passierte Tomaten \\
1 & Bund & Basilikum \\
100 & ml & Oliven\"ol \\
2 & & Zwiebeln \\
4 & Zehen & Knoblauch \\
& & Zimt
}{
Den Tofu mit einer Gabel zerdr\"ucken und in Oliven\"ol scharf von allen Seiten anbraten.
Am Schluss die Zwiebeln und danach den Knoblauch hinzugeben und mit anbraten.
Pfeffern, salzen und ein wenig Zimt dazu geben.
Das Tomatenmark unterr\"uhren.

Wenn alles vermischt ist und der Tofu eine sch\"one Farbe angenommen hat mit dem Rotwein abl\"oschen.
Die passierten Tomaten hinzugeben und alles w\"urzen und einkochen lassen, evtl nachw\"urzen.
Das Basilikum hacken und mitkochen lassen.

Passt gut zu Nudeln, kann aber auch als Bolognese f\"ur die Linsenlasagne verwendet werden.  }

\chapter{S\-o\-ß\-e\-n}
% sossen
\thispagestyle{empty}
%\newpage
\rezept{Bechamelso\ss{}e}{20}{800g; reicht knapp f\"ur eine Linsenlasagne}{
50 & g & Butter \\
50 & g & Mehl \\
600 & g & Gem\"usebr\"uhe \\
100 & g & Schlagsahne \\
50 & g & frisch geriebener Parmesan \\
 & & frisch geriebene Muskatnuss
}{
\label{rec:bechamel}
Die Butter in einem Topf zerlassen und das Mehl darin unter R\"uhren mit einem Holzl\"offel and\"unsten, bis es Blasen wirft und Butter und Mehl sich gut miteinander verbunden haben.
Daf\"ur zun\"achst einen Holzl\"offel nehmen, damit die Mehl-Butter-Mischung nicht im Schneebesen h\"angen bleibt.

Br\"uhe (Zimmertemperatur) nach und nach dazugießen und dabei kr\"aftig mit einem Holzl\"offel r\"uhren, damit keine Kl\"umpchen entstehen.
Unter st\"andigem R\"uhren bei kleiner Hitze etwa 3 Minuten kochen.
Daf\"ur nun den Schneebesen nehmen, damit die So\ss{}e garantiert glatt und sch\"on cremig wird.

Hitze ausschalten, Sahne und den Parmesan unterr\"uhren und die So\ss{}e mit Salz, Pfeffer und frisch geriebenem Muskat abschmecken (Muskat sparsam dosieren!).
Topf mit der So\ss{}e vom Herd nehmen.  }

\rezept{Käsesoße}{20}{
für Nudeln, Windbeutel}{
3     & EL & Butter \\
3     & EL & Mehl \\
1/2   & TL & Salz \\
1/4   & TL & Senf \\
1/8   & TL & Pfeffer \\
1 1/2 & Tasse & Milch \\
1     & Tasse & Cheddar/Käse \\
}{
Eine Mehlschwitze machen: Butter in der Pfanne schmelzen und das Mehl dazu geben.
Zwei Minuten kochen lassen und nach und nach die Milch hinzugeben und zu einer Bechamelsoße verrühren.

Mit den Gewürzen abschmecken und den Käse hineinschmelzen.}

\rezept{Pistazienpesto}{20}{
zum Beispiel zum Grillkartoffelsalat}{
1 & Zehe & Knoblauch \\
3 & EL & Basilikum \\
4 & EL & Pistazienkerne \\
& & ungesalzen \\
5 & EL & Mayonnaise \\
2 & TL & Wei\ss{}weinessig \\
1/2 & TL & grobes Meersalz \\
1/4 & TL & Pfeffer
}{
Den Knoblauch zerkleinern.
Basilikum und Pistazien hinzuf\"ugen und alles fein hacken.
Die Mischung in eine gro\ss{}e Sch\"ussel geben und mit den \"ubrigen Zutaten vermischen.  }

\rezept{Pilzsoße}{20}{
etwa zwei Portionen}{
  300 & g & gemischte Pilze \\
  1/2 & Bund & Petersilie \\
  1 & kleine & Zwiebel \\
  1 & Zehe & Knoblauch \\
  2 & EL & Olivenöl \\
  250 & g & Sahne oder Bechamelsoße \\
  & & Salz und Pfeffer \\
}{
\label{rec:pilzsosse}
Alle Zutaten klein schneiden.

Öl erhitzen, Zwiebeln und Knoblauch hinzugeben.
Pilze portionsweise hinzugeben und anbraten.

Mit Salz und Pfeffer abschmecken und mit der Sahne ablöschen.
15 Minuten köcheln lassen, bis die Pilze weich sind.

Dazu schmecken gut Knödel, siehe Seite \ref{rec:semmelknoedel}.
}

\rezept{
Sesammayo
}{20}{
vegan Mayo
}{
& & Tahin \\
& & Wasser
}{
In einer Pfanne das Tahin anbraten und wie eine Bechamelsoße mit Wasser vermischen.
Ergibt eine Soße ähnlich wie Mayonaise.
}

\rezept{Vegane Mayonnaise}{100}{gehaltvoll}{
150 & ml & "Ol \\
50 & ml & Sojamilch \\
1 & EL & Senf \\
1 & Prise & Salz \\
1 & Prise & Pfeffer \\
$1/2$ && Zitrone
}{"Ol, Sojamilch, Senf und Zitronensaft mit einem Stabmixer
zu einer cremigen Masse p"urieren. Mit Salz und Pfeffer abschmecken.

Tip: "Ol und Sojamilch sollten etwa die gleiche Temperatur haben.}

\chapter{D\-e\-s\-s\-e\-r\-t\-s}
% dessers / nachtische
\thispagestyle{empty}
%\newpage
\input{rezepte/rec_birchermuesli}
\rezept{Brat\"apfel}{20}
{Fast vegan}
{4 & saure &  \"Apfel \\
 & & z. B. Braeburn, Boskoop\\
 & und & \\
100 & g & Rohmarzipan \\
3 & EL & gehackte Mandeln \\
3 & EL & gehackte N\"usse \\
1 & EL & optional Rosinen \\
1 & TL & Zimt \\
& etwas & Butter }
{ Das Kerngeh\"ause der \"Apfel gro\ss{}z\"ugig ausstechen und oben einen Deckel abschneiden.
Die vorbereiteten \"Apfel in eine (gefettete) Backform stellen.

Die F\"ullung mischen und in die \"Apfel f\"ullen.

Im Backofen 30 Minuten bei Umluft 200 Grad Celsius backen.

Kurz abkuehlen lassen, mit Vanilleso\ss{}e anrichten und warm servieren.}

\input{rezepte/rec_indischerbrotpudding}
\input{rezepte/rec_milchreis}
\rezept{Obstquark}{20}{
a la Mama}{
& & Obst/Beeren \\
250 & g & Quark \\
125 & g & Joghurt \\
& & Zucker \\
& & Zitrone
}{ Alles zusammenmischen, zum Schluss Obst oder Beeren hinzugeben.}

\rezept{Panna cotta}{20}{}{
500 & ml & Milch oder Schlagsahne \\
25 & g & Zucker \\
1 & Blatt & Gelatine \\
& & Rum oder Vanille}{
Die Gelatine in kaltem Wasser einweichen.
Die Milch mit Vanille und Zucker aufkochen und vom Feuer nehmen.
Die aufgeweichte Gelatine einrühren und durch ein Sieb geben.
(Vorsicht: Gelatine darf nicht gekocht werden!)
In Formen füllen und über Nacht im Kühlschrank ruhen lassen.
Zum Servieren die Formen 5 Sekunden in warmes Wasser tauchen.}

\chapter{S\-n\-a\-c\-k\-s}
% snacks
\thispagestyle{empty}
%\newpage
\rezept{
Bauernkäse
}{20}{
traditionell englischer farmers cheese
}{
1 & l & frische Milch > 3.5 Prozent \\
1/8 & Tasse & Weißweinessig \\
1/2 & TL & Salz \\
}{
Milch auf 87 Grad Celsius erhitzen, ausstellen, den Essig dazu geben und rühren bis die Milch Klumpen formt.
15 Minuten stehen lassen und durch ein Handtuch geben.
Den Käse auswringen und in einer Schüssel mit Salz und optional mit Kräutern mischen.
}

\input{rezepte/rec_kandierteringwer}
\input{rezepte/rec_maroni}
\input{rezepte/rec_zwiebeln}
\chapter{B\-r\-o\-t\-e}
% brote
\thispagestyle{empty}
%\newpage
\rezept{Sauerteigbrot ultimate}{20}{Tagesprojekt, 2 Brote, 3500 kcal}{
Levain & & (100\% Hydration) \\
35 & g & Starter \\
35 & g & Weizenvollkornmehl \\
35 & g & Weizenmehl \\
70 & g & Wasser, Raumtemperatur \\
Teig & & (80\% Hydration) \\
175 & g & Levain \\
800 & g & Mehl (Mischung) \\
75 & g & Vollkornmehl \\
700 & g & Wasser \\
20 & g & Salz
}{
Brot:

 8:00 am: Levain ansetzen, dafür alle Zutaten mischen und 5-6 Stunden bei etwa 25 C mit einem feuchten Handtuch abgedeckt ruhen lassen.

11:30 am: Teig ansetzen, dafür die Mehle mischen und mit 620g Wasser vermengen. Dann mit einem feuchten Handtuch abgedeckt ruhen lassen.
Hier binden sich die Glutenstraenge, der Teig sollte dabei zu einer weniger klebrigen, elastischeren Masse werden.

 1:00 pm: Wenn der Levain fertig ist (gerade dann wenn die Oberfläche anfängt zu fallen) selbigen zum Teig geben, mit angefeuchteten Fingern Löcher hineindrücken und alles vermischen.
Den Teig mit der Slap-and-fold Technik ohne Mehl bearbeiten bis er nicht mehr so schlimm klebt, dann mit einem feuchten Handtuch abgedeckt 25 min ruhen lassen.
Sauerteig nie kneten, nur falten und schlagen.

Tip: Der Teig klebt weniger an den Haenden wenn man diese vorher ein wenig anfeuchtet.

 1:30 pm: Mit dem Salz und dem restlichen Wasser vermengen und wieder ohne Mehl slap-and-folden.

(Das Salz nicht vorher dazu geben, da wir dem Mehl die Chance geben wollen sich voll Wasser zu saugen.
Der osmotische Effekt vom Salz würde dies stören.)

 1:35 pm: Gehen/Fermentieren lassen bei 25 Celsius.

 1:50 pm: Den Teig in der Schüssel falten: den Teig vom Rand hochziehen und über sich selbst falten, einmal um die Schüssel herum. Nr. 1

 2:05 pm: Falten Nr. 2

 2:20 pm: Falten Nr. 3

 2:50 pm: Falten Nr. 4

 3:20 pm: Falten Nr. 5

 3:50 pm: Falten Nr. 6 und weiterhin ruhen lassen.

 6:00 pm: Auf die ungemehlte Arbeitsfläche geben und mit angefeuchteten Händen und angefeuchtetem Teigschaber in Zwei teilen und ohne zu kneten grob rund formen.

Tip: Der Teig klebt nicht so schlimm am Teigschaber wenn man ihn schnell wegzieht.

Unbedeckt aber oberflächlich gemehlt auf der Arbeitsfläche 20 Minuten ruhen lassen.

 6:20 pm:
Umdrehen und zusammenfalten und mit dem Teigschaber etwas Oberflaechenspannung erzeugen.
In mit großzügig gemehlten Handtüchern ausgelegten Schalen ruhen lassen mit dem Bauch nach oben. (Im Kühlschrank 12-14 Stunden wenn nötig über Nacht.)

 8:20 oder am nächsten Morgen: Den ersten Laib backen.

Poke test: Wenn der Teig fast gänzlich aber nicht komplett zurück springt hat er fertig geruht.
Wenn er ganz zurück springt braucht er mehr Zeit, wenn er gar nicht zurück springt ist es zu spaet.

Dutch Oven bei 260 Celsius vorheizen, Brot mehlen und Dutch Oven mehlen, Laib hineinkippen und leicht asymmetrisch einen halben cm tief einschneiden.
Erst 20 Minuten lang bei 260 Celsius dann 20-30 Minuten 230 Celsius ohne Deckel backen.
Herausholen wenn das Brot eine gute Farbe hat.
Danach den zweiten Laib genauso backen, dafür den Topf 15 Minuten aufheizen lassen.
(Dutch oven - combo cooker - kombi-topf:
Eine Pfanne und ein Topf, gusseisern die bei Bedarf zusammenpassen.)

Für den zweiten Laib zwanzig Minuten ein Blech als Hitzeschirm unter den Topf schieben.

Vorm Anschneiden eine Stunde auf einem Rost oder im halboffenen Ofen auskühlen lassen. }

\rezept{Sauerteigbrot easy}{20}{1 Brot}{
50 & g & aktiven Sauerteigstarter \\
350 & g & lauwarmes Wasser \\
350 & g & Weizenmehl \\
100 & g & Vollkornmehl \\
10 & g & Salz
}{
Das Wasser mit dem Starter verrühren und danach mit dem Mehl und Salz soweit vermischen, dass keine Klümpchen mehr existieren.
Über Nacht bei etwa 21 Grad stehen lassen.
Wenn der Teig auf etwa das doppelte gewachsen ist und unter schütteln leicht wackelt kann er geformt werden und eine Stunde im Kühlschrank warten.
In der Zeit den Ofen vorheizen und backen werden wie das oben beschriebene ultimative Sauerteigbrot.
}

\rezept{Sauerteigstarter}{20}{etwa 170 kcal pro 100g}{
  & & (Roggen-) Vollkornmehl \\
  & & Weizenmehl \\
  & & lauwarmes Wasser (25 C) \\
}{
In einem Halbliterglas mit aufgelegtem Deckel an einem dunklen, etwa 25 Grad warmen Ort lagern.
Das Gewicht des Glases vorher notieren und Mengen mit einer Waage ausmessen.

Tag 1:
150g handwarmes Wasser und
100g Bio-Roggenvollkornmehl mischen bis die Masse keine Klumpen mehr hat. Mit einem lose sitzenden Deckel abdecken und 12 bis 24 Stunden ruhen lassen.

Tag 2+3:
70g Starter,
50g Bio-Roggenvollkornmehl,
50g ungebleichtes Weizenmehl und
115g handwarmes Wasser

Tag 4+5:
70g Starter,
50g Bio-Roggenvollkornmehl,
50g ungebleichtes Weizenmehl und
100g handwarmes Wasser

Tag 6:
50g Starter,
50g Bio-Roggenvollkornmehl,
50g ungebleichtes Weizenmehl und
100g handwarmes Wasser

Tag 7 und alle weiteren:
25g Starter,
50g Bio-Roggenvollkornmehl,
50g ungebleichtes Weizenmehl und
100g handwarmes Wasser

Im Schrank bei etwa 25 Grad lagern: jeden Tag füttern.

Kann im Kühlschrank gelagert werden, dafür 25g Starter mit 100g Mehl und 100ml Wasser füttern und zwei Stunden später in den Kühlschrank stellen.
Gefüttert wird alle 5 bis 7 Tage.

Kann eingefroren werden, dafür ein paar Stunden nach dem Füttern den aktiven Starter einfrieren.

Kann getrocknet werden, dafür auf einem Backpapier ausstreichen und trocknen, zerbröseln und in einem geschlossenen Gefäß lagern.

Tipp: Nicht mehr benötigten Starter mit etwas Wasser und optional einem Ei mischen und in einer geölten Pfanne zu einem Pfannkuchen backen, z.B. mit Sesam und Frühlingszwiebeln oder mit Rosinen und Zucker bestreuen.

Sauerteigstarter ist aktiv wenn er sein Volumen in etwa verdoppelt hat und gerade dabei ist herunterzusinken (das sollte etwa 5-6 Stunden nach dem Füttern der Fall sein, bei 25 Grad).}


%\newpage
\input{rezepte/rec_hefebrot}
\input{rezepte/rec_marraqueta}
\chapter{s\-ü\-ß\-e B\-a\-c\-k\-w\-a\-r\-e\-n}
% suesse backwaren
\thispagestyle{empty}
%\newpage
\rezept{American Pancakes}{20}{
etwa 4 mittelgroße ~}{
1 & Tasse & Mehl \\
1 & & Ei \\
2 & TL & Backpulver \\
1 & TL & Natron \\
1/3 & Tasse & Zucker und Vanille \\
1 & Tasse & Buttermilch \\
 & & optional Blaubeeren \\
 & & am Besten frisch
}{
Zuerst die trockenen Zutaten mischen und dann die nassen dazu geben.
Zu einem z\"ahfl\"ussigen Teig vermischen und in Butter von beiden Seiten bei mittlerer Hitze ausbacken.
Bei Bedarf kann man Blaubeeren (in die Pfanne, direkt nach dem Teig) oder Schokoladenchips (in den Teig) dazu geben.
Mit dem umdrehen warten, bis die Ränder gebacken aussehen und der Teig von den aufsteigenden Blasen Löcher bekommt..
}

\rezept{
  Bananen-Pfannkuchen
}{20}{
  amerikanisches Rezept
}{
  1 & & Banane \\
  2 & & Eier \\
  100 & g & Sauerteigstarter \\
  & & Wasser
}{
  In einer gebutterten Pfanne ausbacken und genießen.
}

\rezept{Mamas Pfannkuchen}{20}{
Reicht für 6 große}{
3 & & Eier \\
200 & g & Mehl \\
200 & ml & Milch \\
1 & Prise & Salz \\
1 & Prise & Zucker
}{
Zutaten zu einemehr fl\"ussigen Teig zusammen mischen und eine Stunde ziehen lassen.

Pfannkuchen in einer gefetteten Pfanne bei mittlerer Hitze (4-5/9) und mit Deckel ausbacken.
Nach Belieben mit Obst oder Gem\"use belegen, ein halber Apfel pro Pfannkuchen.}

%\newpage
\rezept{Profiteroles/Windbeutel}{20}{
}{
250 & ml & Wasser \\
115 & g & Butter \\
3 & g & Salz (1/2 TL) \\
7 & g & Zucker \\
125 & g & Mehl \\
4 & & Eier \\
}{
Das Wasser in einem Topf auf mittlerer Hitze geben.
Die Butter würfeln und im Wasser schmelzen.
Zucker, Salz und Mehl mit einem Holzlöffel untermischen und den Teig konstant rühren bis sich auf dem Boden des Topfes ein Film bildet.
Zwei bis drei Minuten kochen.
Von der heißen Platte nehmen und drei Minuten auskühlen lassen, dann die Eier nacheinander einzeln dazu geben und verrühren.
Der Teig sollte glatt und nicht klebrig sein und eine Spitze halten können.

Mit einem Sprizbeutel in etwa 3.5 bis 4 cm große Kreise spritzen.
Den Finger in Wasser anfeuchten und die Spitzen herunter drücken, sodass sie nicht verbrennen.
Optional mit Eigelb bestreichen.
Bei 190 Grad Ober-Unterhitze etwa 25 - 30 Minuten backen.

Wichtig: dabei den Ofen nicht öffnen, sie fallen sonst zusammen.

Kann mit etwa 700 ml Crema pastelera gefüllt werden oder mit geschlagener Sahne, vielleicht auch mit Käsesoße.
Füllung entweder von unten hineinspritzen oder aufschneiden und füllen.}

\chapter{K\-u\-c\-h\-e\-n}
% kuchen
\thispagestyle{empty}
%\newpage
\rezept{Apfel Muffins}{20}{
etwa 16 Stück}{
250 & g & Mehl \\
150 & g & Zucker \\
125 & g & Butter \\
2 & & Eier \\
1 & Pkg. & Vanillezucker \\
1/2 & Pkg. & Backpulver \\
250 & ml & Milch \\
2 & & Äpfel (z.B. Booskoop)
}{
Zutaten wie Mehl, Zucker, Margarine, Eier, Vanillezucker, Backpulver und Milch in eine große Schüssel geben und mit dem Mixer durchrühren.

Danach die geschälten, in kleine Stücke geschnittenen Äpfel hinzugeben und in den Teig mischen.

Die Papierförmchen jeweils zu 3/4 mit dem Teig füllen und auf ein Backblech stellen.

Im vorgeheizten Backofen bei 150°C Heißluft ca. 20-25 Minuten backen.}

\input{rezepte/rec_banoffee}
\rezept{Bisquit}{20}{sehr fluffy, gut zu schneiden, 1378 kcal}{
5 & & Eier \\
125 & g & Zucker \\
75 & g & Mehl \\
25 & g & Maismehl
}{
Eier und Zucker über einem Wasserbad vermischen bis der Zucker sich aufgelöst hat, das Ei soll nicht stocken.
Vom Wasserbad nehmen und mit der Maschine aufschlagen.
Gesiebte Mehl-Kakao-Mischung vorsichtig unterheben und im Ofen bei 160 Grad etwa eine halbe Stunde backen.

Fingertest: der Bisquit ist fertig wenn der Fingerabdruck verschwindet.

Notiz: Für Schokobisquit 30g Kakao und 170g Zucker verwenden: 1659 kcal.}

\rezept{Blondies}{20}{}{
230 & g & Butter \\
340 & g & brauner Zucker \\
& & Vanille \\
2 & & Eier \\
280 & g & Mehl \\
1/2 & TL & Natron \\
1 & TL & Backpulver \\
1 & Prise & Salz \\
6 & EL & getrocknete Cranberries \\
240 & g & weiße Schokolade \\
}{
Den Ofen auf 180 Celsius vorheizen, eine quadratische, 23cm große Kuchenform mit Butter fetten und mit Backpapier auslegen.
Die Butter in einem Topf schmelzen und mit dem Zucker und Salz vermengen, etwas schaumig schlagen.
Die Vanille unterrühren und noch einmal schaumig schlagen.
Den Topf von der Hitze nehmen.
Kurz abkühlen lassen und dann die Eier in die Buttermischung geben und gut umrühren.
Das Mehl, Natron und Backpulver sieben und nach und nach unterrühren, sodass keine Klumpen entstehen.
Soweit abkühlen lassen dass die Schokolade nicht schmilzt und dann selbige zusammen mit den Cranberries unterheben.
In die vorbereitete Form geben und gleichmäßig verteilen.

Etwa 35 bis 40 Minuten backen bis die Kanten fest und die Mitte noch ein bisschen weich ist.
Mindestens 10 Minuten auf dem Rost auskühlen lassen und in Quadrate schneiden.

Die fertigen Blondies halten sich bis zu einer Woche in einer luftdichten Dose.  }

%\newpage
\input{rezepte/rec_bananabread_with_dates}
\input{rezepte/rec_bananenbrot}
\rezept{
  Hermann
}{20}{
  Kuchen - 3000 kcal
}{
200 &     g & Portion Hermann-Teig von Tag 10 \\
200 &     g & Weizenmehl \\
100 &     g & Zucker \\
200 &    ml & Milch \\
100 &    ml & Öl \\
  3 &       & Eier \\
  1 &  Pck. & Vanillezucker \\
  1 & Prise & Salz \\
  2 &    TL & Backpulver \\
100 &     g & optional: geriebene oder gehackte Nüsse, Mandeln, Rosinen oder Schokostreusel \\
}{
Alle Zutaten zu einem glatten Teig verrühren.
In eine gefettete, bemehlte Kastenform oder Gugelhupfform füllen.

Im vorgeheizten Backofen bei 180 °C (Umluft 160 °C) circa 45 bis 55 Minuten backen.
Zum Ende der Backzeit eine Stäbchenprobe machen.
}

\rezept{
Pflegeanleitung für Hermann-Teig
}{20}{
  }{
100 &  g & Weizenmehl \\
150 &  g & Zucker \\
150 & ml & Milch \\
}{
Egal ob geschenkt oder nach oben stehender Anleitung selbst hergestellt – von nun an wird der Hermann-Teig im Kühlschrank aufbewahrt.

Damit sich der Hermann-Teig vermehrt und schließlich zum Backen verwendet werden kann, wird er so gepflegt:

\begin{itemize}
\item  1. Tag: Ruhen lassen.
\item  2. Tag: Umrühren.
\item  3. Tag: Umrühren.
\item  4. Tag: Umrühren.
\item  5. Tag: Füttern – 50 g Weizenmehl, 75 g Zucker und 75 ml Milch zugeben. Gut verrühren.
\item  6. Tag: Umrühren.
\item  7. Tag: Umrühren.
\item  8. Tag: Umrühren.
\item  9. Tag: Umrühren.
\item 10. Tag: Füttern – 50 g Weizenmehl, 75 g Zucker und 75 ml Milch zugeben. Gut verrühren.
\end{itemize}
}

\rezept{
  Kaiserschmarrn
}{20}{
  eine Portion
}{
 15 & g     & Rosinen \\
  1 & EL    & Rum \\
  2 & Eier  & getrennt \\
 15 & g     & Zucker \\
1/2 & Prise & Salz \\
1/2 & Pck   & Vanillezucker \\
180 & ml    & Milch \\
 65 & g     & Mehl \\
 20 & g     & Butter \\
    &       & Puderzucker \\
}{
Rosinen 30 Minuten mit Rum in einer Schüssel einweichen.

Eigelb, Zucker, Salz und Vanillezucker in einer Schüssel schaumig rühren, bis die Masse hellgelb cremig ist.
Erst die Milch und dann nach und nach das Mehl unterrühren, dann die Rosinen zugeben.
Eiweiß sehr steif schlagen und vorsichtig unter den Teig heben.

Butter in einer Pfanne erhitzen, Teig einfüllen und bei kleiner Hitze braten, bis die Unterseite leicht gebräunt ist.
Dabei immer wieder wenden, bis alles leicht angebraten ist und zerreißen.

Anrichten und mit Puderzucker bestreuen.
Als Beilage Zwetschgen- oder Apfelkompott dazu servieren.
}

%\newpage
\input{rezepte/rec_blueberrypie}
\rezept{Crema pastelera}{20}{~925 kcal}{
250 & ml & Milch \\
3 & & Eigelb \\
2 & EL & Maisstärke \\
3 & EL & Zucker \\
2 & EL & Butter \\
& & Vanille \\
}{
Die Milch mit ein wenig des Zuckers erhitzen.
In einer Schüssel Eigelb, Stärke, Zucker und Vanille vermischen.
Zum temperieren etwa die Hälfte der Milch dazugeben und mit einem Schneebesen schlagen.
Alles zurück zu der restlichen Milch in den Topf geben, aufkochen und durchgängig schlagen bis die Creme andickt.
Durch ein Sieb in eine Schüssel geben und mit der Butter verrühren.

Abgedeckt auskühlen und ruhen lassen.}

\rezept{Hefeteig}{20}{
ein Blech, f\"ur Blechkuchen}{
400 & g & Mehl \\
1 & Pck. & Hefe \\
1 & TL & Zucker \\
1 & Tasse & (warme) Milch \\
& & oder Wasser \\
6 & EL & \"Ol \\
1 & TL & Salz
}{ Alles zusammen mischen und eine halbe Stunde an einem warmen Ort gehen lassen.

Auf dem Blech ausrollen und mit Obst und Streuseln belegen.
25 bis 30 Minuten bei 175 Grad Umluft backen.}

\rezept{Hefezopf mit Rosinen}{20}{
zu Ostern}{
500 & g & Mehl\\
& & Hefe \\
1 & Pck & Vanillezucker \\
100 & g & Zucker \\
100 & g & Butter \\
1 & Prise & Salz \\
2 & & Eier \\
80 & g & Rosinen \\
125 & g & Milch
}{
Alles zusammen mischen, eine Stunde gehen lassen.
Zopf flechten und mit Milch bestreichen, eine halbe Stunde gehen lassen.

Eine halbe Stunde bei 175 Grad Umluft backen. }

\rezept{Joghurt Muffins}{20}{
etwa 12 Stück a 130 kcal}{
200g & (1 cup) & Joghurt \\
150g & (1 1/2 cup) & Mehl \\
2 Msp. & & Backpulver \\
170g & (1 cup) & Zucker \\
60g & (1/3 cup) & Öl \\
2 & & Eier \\
1 Prise & & Salz \\
  & & Füllung (z.B. Blaubeeren)
}{
Info: 1 cup = 225 ml.

Trockene Zutaten mischen,
Nasse Zutaten mischen, Zucker dazu, Mehl und Backpulver dazu, Beeren/Schokolade dazu, alles zusammen rühren.

Teig in die Muffinformen füllen und bei 180°C etwa 40 bis 50 min backen.}

\rezept{
Sandkuchen
}{20}{
Z.B. mit Apfel
}{
250 &     g & Butter, Raumtemperatur \\
200 &     g & Zucker \\
  4 &       & Eier \\
250 &     g & Mehl \\
  2 &    TL & Backpulver \\
  1 & Prise & Salz \\
  6 &       & Äpfel \\
}{
Die Äpfel schälen und in Stücke schneiden.

Butter, Zucker, Eier mit dem Handmixer schaumig schlagen.

Mehl, Salz, Backpulver vermischen und kurz unterrühren, dann die Apfelstücke unterheben.

Eine Backform fetten und den Teig hineingeben, bei etwa 175 Grad Celsius circa 60 Minuten backen.
}

%\newpage
\input{rezepte/rec_apfelruehrkuchen}
\input{rezepte/rec_rhabarberpie}
\input{rezepte/rec_karottenmuffins}
\rezept{Streusel}{20}{
f\"ur 1 Blech}{
200 & g & Mehl \\
100 & g & Zucker \\
100 & g & Butter \\
1 & Pck & Vanillezucker \\
1 & Prise & Salz
}{Zusammenmischen bis es kr\"umelt und über Obstkuchen geben oder für Plätzchen verwenden.}

%\newpage
\input{rezepte/rec_tortapompadour}
\rezept{Tres Leches}{20}{}{
1/2 & Tasse & Milch \\
1 & Prise & Salz \\
6 & & Eier \\
1 1/2 & Tassen & Zucker \\
2 & Tasse & Mehl \\
3 & TL & Backpulver \\
 & & Vanilleextrakt \\
1 & Tasse & gezuckerte Kondensmilch \\
1 & Tasse & Kondensmilch \\
1 & Tasse & Milch \\
}{
Form einfetten.
Mehl und Backpulver dreimal sieben.
Das Eiweiß zu Eischnee schlagen und das Salz hinzufügen.
Nach und nach den Zucker dazu geben und dann die Eigelbe eins nach dem anderen dazu geben.
Ein Drittel des Mehls darunterheben und Milch und Vanille dazu geben.
Den Rest des Mehl hinzufügen.
Bei 160 Grad eine halbe Stunde backen bis goldenbraun.
In einer Tasse die drei Milch mischen (zimmerwarm).
Den Bisquit aus dem Ofen nehmen, und die Oberfläche mit einem Zahnstocher einstechen.
Mit der Milchmischung nach und nach den Bisquit durchnässen.
Im Kühlschrank auskühlen lassen.
Nach Geschmack mit Crema Pastelera, Merengue, gerösteten Mandeln dekorieren und servieren.
}

%\newpage
\rezept{Sticky Toffee Pudding}{20}{
Puddings - 2500 kcal, Sauce - 1200}{
für & die & Küchlein \\
350 & g & getrocknete Datteln \\
600 & g & Wasser \\
2 & TL & Natron \\
100 & g & Butter \\
350 & g & Zucker \\
4 & & Eier \\
350 & g & Mehl \\
 & & \\
für & die & Soße \\
300 & g & Schlagsahne \\
50 & g & brauner Zucker \\
2 & EL & Melasse \\
}{
Datteln klein hacken und im Wasser unter Rühren weich kochen.
Natron dazu geben und rühren, die Datteln sollten sich auflösen.
Mischung abkühlen lassen

Nun Butter und Zucker schaumig rühren, dann die Eier dazu geben und gemeinsam schaumig rühren.

Dattelcreme dazu geben und mischen, Mehl unterheben.

Muffinformen oder tiefe Backform buttern und mehlen.

Bei 170 Grad Celcius etwa 20 bis 30 Minuten backen.

Für die Soße die Sahne mit dem Zucker aufkochen und dann den Sirup hinzugeben.
Rühren und abkühlen lassen.
}

\rezept{Vegane Apfel Karotte Nuss Muffins}{20}{
24 Muffins a 240 kcal}{
300 & g & Karotten \\
200 & g & Apfel \\
150 & g & Wasser \\
10 & EL & Kokosöl \\
300 & g & Zucker \\
400 & g & geriebene Nüsse \\
300 & g & Mehl \\
2 & EL & Backkakao \\
1 & Pkg & Backpulver \\
1 & Prise & Salz
}{
Karotten und Apfel reiben und mit dem Wasser vermischen.
Kokosöl und Zucker vermischen und zu der Apfel-Karotten-Mischung geben.
Die restlichen Zutaten mischen und darunterheben.

Bei 180 Grad Ober- Unterhitze etwa 20-40 Minuten backen (Kuchen braucht länger als Muffins.)
}

%\newpage
\rezept{Zitronen Polenta Kuchen}{20}{
Für eine 25cm Springform; glutenfrei; 4700 kcal}{
Kuchen & & \\
200   & g     & Butter \\
200   & g     & gemahlene Mandeln \\
200   & g     & Zucker \\
 50   & g     & feiner Maisgries \\
 50   & g     & Maismehl \\
3     & große & Eier \\
      &       & Zeste von 3 Zitronen \\
      &       & Saft einer Zitrone \\
1 1/2 & TL    & Backpulver \\
      & & \\
Sirup & & \\
    &   & Zeste einer Zitrone \\
    &   & Saft von 3 Zitronen \\
125 & g & Puderzucker \\
}{
  % anleitung
Mit Butter eine Springform mit 25cm Durchmesser fetten.
Den Ofen auf 170 Grad Celsius vorheizen.

Zucker und Butter in einem Mixer mixen bis die Masse weiß ist, dann die Zitronenzeste hinuntemischen.

Die Mandeln, den Maisgries und das Backpulver zusammen mischen.

Abwechselnd einen Teil der trockenen Zutaten und die Eier zu der Butter-Zucker-Mischung geben und gut vermixen.

Zum Schluss den Zitronensaft (einer Zitrone) dazu geben und die Mischung in die Springform geben.

Den Kuchen 25 bis 30 Minuten backen, und mit einem Blech abdecken wenn er vor Ende der Backzeit zu braun wird und auf 160 Grad herunter drehen.

Für den Sirup die Zutaten mischen und aufwärmen.

Nach Ende des Backens mit Zahnstochern Löcher in den Kuchen stechen und den Sirup darüber geben, dabei alles verbrauchen.

Abwandlungen:
Mandeln durch Maisgries ersetzen; Zitronen durch Orangen ersetzen. }


\chapter{K\-e\-k\-s\-e}
% kekse
\thispagestyle{empty}
%\newpage
\input{rezepte/rec_alfajores}
\rezept{Chocolate-Chip-Cookies}{20}{
24 Kekse, jeder 160 cals}{
1/2 cup & 115 g & butter, handwarm \\
1/2 cup & 100 g & brauner Zucker \\
1/2 cup & 100 g & Zucker \\
1 & & großes Ei, handwarm \\
1 & & TL vanilla extract \\
1 ¼ cup & 185 g & Mehl \\
1 Tbsp & 7.5 g & Stärke \\
1/2 tsp & 2.5 g & Natron \\
1/2 tsp & 2.5 g & Salz \\
1 1/2 cup & 262 g & backstabile Chocolate Chips \\
1 cup & 100 g & Nüsse (Walnuss, Macadamia...)
}{
Die Butter, den Zucker vermischen, das Ei und die Vanille dazugeben und gut verrühren.
In einer separaten Schüssel das gesiebte Mehl mit der Stärke, dem Natron und Salz vermischen und nach und nach zu der Buttermischung geben und verrühren.
Die Chocolate Chips und Nüsse dazugben und vermischen.
Zu kleinen Bällen formen (etwa 2 TL) und für mindestens eine Stunde im Kühlschrank ruhen lassen.

Den Ofen auf 162 Grad Celsisus vorheizen und Kekse mit etwa 4 cm Abstand auf einem Blech etwa 15-18 Minuten backen bis die Ränder golden werden.
Die Kekse auf dem Blech abkühlen lassen (sie sind jetzt sehr weich und härten beim Abkühlen aus).

Notiz: Statt die Bälle zu backen kann man sie auch einfrieren und Tage später backen, dabei sollte beachtet werden dass sie etwa 20 Minuten bei Raumtemperatur antauen sollten bevor sie gebacken werden. }

%\newpage
\rezept{Lebkuchen}{20}{
f\"ur ein großes oder zwei kleine Lebkuchenh\"auschen}{
175 & g & Honig \\
50 & g & Zucker \\
50 & g & Butter \\
300 & g & Mehl \\
1 & TL & Backpulver \\
1 & EL & Kakao \\
1 & TL & Pfefferkuchengew\"urz \\
2 & & Eiweiss \\
450 & g & Puderzucker
}{
Honig, Zucker und Butter zusammen im Topf erhitzen bis der Zucker sich gel\"ost hat.
Aufpassen, dass der Zucker nicht karamellisiert, Masse abk\"uhlen lassen!

Mehl, Backpulver, Kakao und Pfefferkuchengew\"urz mischen und unter die erkaltete Honigmasse r\"uhren.
Teig eine Stunde ruhen lassen.

Formen ausstechen und bei Umluft 180 Grad, 10-15 min backen.

Zum Zusammenkleben Eiwei\ss{} oder Kichererbsenwasser steif schlagen und Puderzucker unterr\"uhren.
Für die Dekoration kann man den Puderzucker auch mit Wasser und oder Zitronensaft anrühren, die Version mit Eiweiss oder Kichererbsenwasser wird aber stabiler.

Kalte Pl\"atzchen mit Zuckerguss und Gummib\"archen, Mandeln, N\"ussen verzieren.
}

\rezept{Streuselkekse}{20}{
f\"ur 1 Blech}{
200 & g & Mehl \\
100 & g & Zucker \\
100 & g & Butter \\
1 & Pck & Vanillezucker \\
1 & Prise & Salz \\
2-3 & EL & Pflaumenmuß
}{Streusel zusammenkneten und am Ende Marmelade hinzuf\"ugen, sodass eine feuchte, klebrige Masse entsteht.
Zu kleinen Kugeln formen und aufs Blech drücken, braucht kein Backpapier.

15-20 Minuten bei 170 Grad backen, bis die Kekse ein wenig gebräunt sind.}

\rezept{Zimtsterne}{20}{
vegan}{
300 & g & Puderzucker \\
2 & EL & Zimt \\
8 & EL & Wasser \\
1 & EL & Zitronensaft \\
150 & g & gehackte Mandeln \\
200 & g & gehackte Haseln\"usse \\
1 & EL & Orangenschale
}{
Vorsicht, der Teig klebt.
Etwa 10 Minuten backen.  }

%\newpage
\rezept{
Berliner Brot
}{20}{
Ein Klassiker
}{
   2 &     &                       Eier \\
   2 &  EL &              heißes Wasser \\
 250 &   g &                     Zucker \\
  65 &   g &                      Sirup \\
   1 &  Fl &                   Rumaroma \\
   1 & Msp &                     Nelken \\
   1 &  EL &                       Zimt \\
  65 &   g & geriebene Bitterschokolade \\
 250 &   g &                       Mehl \\
   1 &  TL &                 Backpulver \\
 250 &   g &           ganze Haselnüsse \\
 100 &   g &                Puderzucker \\
}{
Eier mit Wasser schaumig schlagen und den Zucker nach und nach dazu geben bis eine cremige Masse entsteht.

Sirup, Rumaroma, Nelken, Zimt und die geriebene Schokolade dazu geben.

Mehl und Backpulver dazu geben.

Haselnüsse unterheben.

Die Masse auf einem Backblech etwa einen halben Zentimeter dick streichen.

Bei 200 Grad Celsius für 20 Minuten backen und direkt in der heißen Masse kleine Stangen (1.5 cm x 4 cm) schneiden.
Dann mit in 1-2 EL Wasser gelöstem Puderzucker bestreichen und abkühlen lassen.
}

\input{rezepte/rec_mandelhoernchen}
\rezept{
Vanille Kipferl
}{20}{
Zubereitung ein bisschen kompliziert
}{
300 &   g &                         Mehl \\
250 &   g &                 kalte Butter \\
100 &   g & gemahlene Mandeln oder Nüsse \\
100 &   g &             (Vanille) Zucker \\
  1 & Pse &                optional Zimt \\
  1 & Pse &                         Salz \\
}{
Die Zutaten rasch zu einem Mürbeteig verkneten und münzdicke Rollen formen. Diese 30 Minuten kühl stellen.

Von den Rollen etwa fingerdicke Scheiben abschneiden und in kleine Hörnchen formen und auf ein gebuttertes Blech legen.

Bei 180 Grad Celsius mit Ober/Unterhitze etwa 12 Minuten backen, nach dem Herausholen in Vanillezucker/Puderzuckergemisch wälzen und dann abkühlen lassen.
}

%\newpage
\input{rezepte/rec_stutenkerle}
\chapter{M\-a\-r\-m\-e\-l\-a\-d\-e\-n}
% marmeladen
\thispagestyle{empty}
%\newpage
\input{rezepte/rec_birnenmarmelade}
\chapter{G\-e\-t\-r\-ä\-n\-k\-e}
% getraenke
\thispagestyle{empty}
%\newpage
\rezept{Chaitee}{20}
{Gew\"urztee}{
& & Ingwer \\
& & Nelken \\
& & Pfeffer \\
& & Zimt \\
& & Muskatnuss \\
& & Kardamom \\
& & schwarzer Tee \\
& & Milch
}{Gew\"urze in einem Topf mit Milch und Wasser kochen, durch ein Sieb gie\ss{}en und hei\ss{} servieren.}

\input{rezepte/rec_ingwerbier}
\rezept{Zitronenlimo}{20}{alla Fari}{
1.5 & l & Wasser \\
6 & EL & Rohrzucker \\
6 & Staengel & frische Minze \\
2 && Zitronen
}{
Die Minze im Wasser etwa 5 Minuten auskochen und danach herausnehmen.
In der Zwischenzeit die Zitronen auspressen.
Den frischen Zitronensaft mit Fruchtfleisch sowie den Rohrzucker in das hei\ss{}e Minzwasser geben und umr\"uhren.
Abkuehlen lassen und kalt stellen.

Zum Servieren einen Teil Limonade mit zwei Teilen Wasser in ein Glas geben und nach Gutd\"unken Dekorieren.}

\chapter{M\-a\-ß\-e}
%\newpage
%\input{masse}


\end{document}
