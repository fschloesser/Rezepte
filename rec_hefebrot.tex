\rezept{Hefebrot}{20}{dauert mind. 4 Stunden}{
500 & g & Mehl \\
1 & TL & Salz \\
10 & g & Trockenhefe \\
320 & g & Wasser \\
& & optional Zwiebel, Kerne, \\
& & getrocknete Tomate, \\
& & Rosinen etc.
}{
Das Mehl, Salz und die Hefe mischen, mit dem Wasser vermengen.
Es mag sein, dass die Mischung zu trocken oder zu flüssig erscheint, dies sollte nicht der Fall sein.
Auf einer gemehleten Oberfläche 10 bis 15 Minuten zu einem elastischen Teig verkneten der nicht mehr an der Arbeitsfläche klebt.
Mit einer dünnen Schicht Öl bedecken und an einem warmen Ort (unter 55 Grad, zum Beispiel auf einer handwarmen Heizung) in einer mit einem Küchentuch bedecketen Schüssel 1 bis 2 Stunden (oder bis sich das Volumen verdoppelt hat) ruhen lassen.
Aus der Schüssel nehmen und nochmal 1 bis 2 Minuten kneten, dabei die handtuchtrockenen Extras hinzufügen.
In eine geölte Backform geben oder in der gewünschten Form auf ein Blech legen und nochmal an einem warmen Ort 1/2 bis 1 Stunde (oder bis sich das Volumen verdoppelt hat) gehen lassen.
Im 230 Grad Celsius vorgeheizten Ofen etwa 20 bis 40 Minuten backen (je nach Form).
Wenn die Oberfläche golden wird das Brot aus dem Ofen nehmen und vor dem Anschneiden auf dem Rost 10 bis 15 Minuten auskühlen lassen.

Notiz: Die wichtigen zwei Dinge sind, dass der Teig gut geknetet wird um die Glutenstränge zu entwickeln und dass er lange genug ruht, denn in dieser Phase entwickelt sich der Geschmack. }
