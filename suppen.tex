%\rezept{Titel}{20}{Untertitel}{150 & ml & Ei \\3 & Laib & Brot }{Ei aufs Brot spiegeln und mit Seffer und Pfalz bestreuen.}

\rezept{UNGETESTET - Dal Shorba}{20}{Ein Topf voll Vegetarische Linsensuppe}{
2 & & Zwiebeln \\
2 & & Knoblauchzehen \\
1/2 & Knollen & Ingwer \\
2 & gr\"une & Chilischoten \\
4 & EL & Sonnenblumen\"ol \\
4 & TL & Kurkuma \\
4 & TL & Kreuzk\"ummel \\
1.3 & Liter & Gem\"usebr\"uhe \\
225 & g & rote Linsen \\
1 1/2 & & Zitronen \\
175 & ml & s\"u\ss{}e Sahne \\
200 & g & Basmatireis \\
75 & g & Dinkel 
}{
Zwiebeln, Knoblauch, Ingwer und Chili hacken.

Sonnenblumen\"ol in einen Topf geben und Kurkuma und Kreuzk\"ummel dazu geben. 
Das ganze kurz anbraten sodass es leicht braun wird, dann das gehackte Gem\"use dazu geben.

Sobald die Zwiebeln glasig sind mit der Gem\"usebr\"uhe aufgie\ss{}en und die Linsen dazu geben und alles langsam weich kochen. 

Dann die Suppe p\"urieren und mit dem Zitronensaft und der Sahne abschmecken.

In einem anderen Topf den Reis und den Dinkel mit einem halben Liter Wasser k\"ocheln lassen, bis keine Fl\"ussigkeit mehr auf dem Boden steht. 
Portionsweise die Suppe mit einer Kugel Reis garnieren und servieren.  }

\newpage

\rezept{Gazpacho}{20}{F\"ur hei\ss{}e Tage und vier Portionen}{
1 & kg & Tomaten \\
4 & Zehen & Knoblauch \\
1 & & Gem\"usezwiebel \\
2 & kleine & Paprikaschoten \\
10 & EL & Balsamico- oder Weinessig \\
6 & EL & Oliven\"ol \\
4 & Scheiben & Toastbrot \\
1 & kl. Dose & Tomaten \\
1/2 & l & Br\"uhe \\
 & & Salz und Pfeffer
}{
Das Gem\"use putzen und in St\"ucke schneiden (die Tomaten brauchen nicht gesch\"alt zu werden.
Alle Zutaten werden im Mixer p\"uriert, am Besten in mehreren Partien, wobei jedesmal etwas Br\"uhe gegeben werden muss.
Auch das Toastbrot wird mit p\"uriert, es dient der Bindung.
Am Schluss l\"asst man das \"Ol mit in den Mixer flie\ss{}en.

In einer gro\ss{}en Sch\"ussel alles gut verr\"uhren und f\"ur mindestens eine Stunde im K\"uhlschrank gut durchk\"uhlen lassen.

Mit frischem Baguette an hei\ss{}en Tagen ein Hochgenu\ss{}. 

Tip: Wer mag, kann in kleine W\"urfel geschnittene Tomate, Gurke und Zwiebel separat dazu reichen.
Die Suppe eignet sich hervorragend zum Einfrieren.  }

\rezept{K"urbiss"uppchen}{20}{Vegetarisch}
	{1 && Hokaido-K"urbis \\
	1 && Kartoffel \\
	1 && Knoblauchzehe \\
	1 && Zwiebel \\
	1 && Paprika (rot) \\
	1 && Zucchini \\
	1 && Kokosmilch \\
	&& Br"uhe, Salz, Pfeffer}
{K"urbis und Kartoffel (sehr) grob W"urfeln und in einem
passenden Topf leicht anbraten. Zwiebeln und Knoblauch ebenfalls
grob gew"urfelt dazu geben und kurz mitbraten. Mit Br"uhe aufgie"sen
und weichkochen. Alles p"urrieren. Paprika und Zucchini W"urfeln und
in die Suppe geben. Mit Kokos-Milch aufgie"sen und mit Salz und
Pfeffer abschmecken.}
