%\rezept{Titel}{20}{Untertitel}{150 & ml & Ei \\3 & Laib & Brot }{Ei aufs Brot spiegeln und mit Seffer und Pfalz bestreuen.}

\rezept{Gemüsesuppe}{20}{traditional}{
2 & & Kartoffeln \\
4 & & Möhren \\
1/2 & Stange & Lauch \\
100 & g & Nudeln \\
100 & g & Linsen \\
& viel & Petersilie
}{
Gemüse klein schneiden und mit den Nudeln und den Linsen in einen Topf geben.
Mit Wasser auffüllen bis alles großzügig bedeckt ist.
Mit Salz, Pfeffer und Muskat würzen und kochen bis das Gemüse gar ist.}

\rezept{Linseneintopf}{20}{simpel, 4 Portionen}{
1/2 & & Kürbis \\
300 & g & Linsen (rot, gelb) \\
& & Curry \\
& & Koriander \\
& & Ingwer \\
& & Rosinen \\
& optional & Petersilie
}{
Die Linsen waschen, den Kürbis in Würfel schneiden und beides im Topf kurz anbraten und mit Wasser ablöschen, sodass das Gemüse fast bedekt ist.
Salz, Pfeffer und Curry (nicht Goldelefant) hinzugeben (nach Geschmack Ingwer) und 10-15 Minuten weichkochen.
Mit einem Kartoffelstampfer den Kürbis ein wenig zerdrücken aber keinen Brei machen.
Nach Gusto mit Koriander und oder Rosinen servieren.
}

\rezept{Gazpacho}{20}{F\"ur hei\ss{}e Tage und vier Portionen}{
1 & kg & Tomaten \\
4 & Zehen & Knoblauch \\
1 & & Gem\"usezwiebel \\
2 & kleine & Paprikaschoten \\
6 & EL & Oliven\"ol \\
4 & Scheiben & Toastbrot \\
1 & kl. Dose & Tomaten \\
1/2 & l & Br\"uhe oder Wasser \\
 & & Salz und Pfeffer
}{
Das Gem\"use putzen und in St\"ucke schneiden (die Tomaten brauchen nicht gesch\"alt zu werden.
Alle Zutaten werden im Mixer p\"uriert, am Besten in mehreren Partien, wobei jedesmal etwas Br\"uhe gegeben werden muss.
Auch das Toastbrot wird mit p\"uriert, es dient der Bindung.
Am Schluss l\"asst man das \"Ol mit in den Mixer flie\ss{}en.

In einer gro\ss{}en Sch\"ussel alles gut verr\"uhren und f\"ur mindestens eine Stunde im K\"uhlschrank gut durchk\"uhlen lassen.

Mit frischem Baguette an hei\ss{}en Tagen ein Hochgenu\ss{}.

Tip: Wer mag, kann in kleine W\"urfel geschnittene Tomate, Gurke und Zwiebel separat dazu reichen.
Die Suppe eignet sich hervorragend zum Einfrieren.  }
