%\rezept{Titel}{20}{Untertitel}{150 & ml & Ei \\3 & Laib & Brot }{Ei aufs Brot spiegeln und mit Seffer und Pfalz bestreuen.}

\rezept{Dinkelpl\"atzchen}{20}{
simpel}{
1 & Bio & Orange \\
175 & g & Puderzucker \\
200 & g & Butter \\
1 & & Eigelb \\
1 1/2 & EL & Milch \\
400 & g & Dinkelmehl
}{
Teig vor dem Ausstechen eine Stunde im K\"uhlschrank stehen lassen.
Etwa 5 Min backen.  }

\rezept{Streuselkekse}{20}{
f\"ur 1 Blech}{
200 & g & Mehl \\
100 & g & Zucker \\
100 & g & Butter \\
1 & Pck & Vanillezucker \\
1 & Prise & Salz \\
2-3 & EL & Pflaumenmuß
}{Streusel zusammenkneten und am Ende Marmelade hinzuf\"ugen, sodass eine feuchte, klebrige Masse entsteht.
Zu kleinen Kugeln formen und aufs Blech drücken, braucht kein Backpapier.

15-20 Minuten bei 170 Grad backen, bis die Kekse ein wenig gebräunt sind.}

\rezept{Zimtsterne}{20}{
vegan}{
300 & g & Puderzucker \\
2 & EL & Zimt \\
8 & EL & Wasser \\
1 & EL & Zitronensaft \\
150 & g & gehackte Mandeln \\
200 & g & gehackte Haseln\"usse \\
1 & EL & Orangenschale
}{
Vorsicht, der Teig klebt.
Etwa 10 Minuten backen.  }

\rezept{Alfajores}{20}{
für etwa 20}{
75 & g & Margarine \\
50 & g & Puderzucker \\
50 & g & Mehl \\
175 & g & Speisest\"arke \\
1/2 & TL & Backpulver \\
1 & & Ei
}{
Margarine mit Puderzucker verrühren.
Das Ei hinzufügen und verrühren.
Mehl, Speisestärke, Backpulver dazu mischen, im Kühlschrank ruhen lassen.

Ausrollen und ausstechen.

10-12 Minuten bei 160-180 Grad backen.

Mit Marmelade/Manjar/Schokolade zu Doppelkeksen zusammenkleben.

Tip: Etwas Mehl kann durch Backakao ersetzt werden.}

\newpage

\rezept{Lebkuchen}{20}{
f\"ur ein Lebkuchenhaus}{
175 & g & Honig \\
50 & g & Zucker \\
50 & g & Butter \\
300 & g & Mehl \\
1 & TL & Backpulver \\
1 & EL & Kakao \\
1 & TL & Pfefferkuchengew\"urz \\
2 & & Eiweiss \\
450 & g & Puderzucker
}{
Honig, Zucker und Butter zusammen im Topf erhitzen bis der Zucker sich gel\"ost hat.
Aufpassen, dass der Zucker nicht karamellisiert, Masse abk\"uhlen lassen!

Mehl, Backpulver, Kakao und Pfefferkuchengew\"urz mischen und unter die erkaltete Honigmasse r\"uhren.
Teig eine Stunde ruhen lassen.

Formen ausstechen und bei Umluft 180 Grad, 10-15 min backen.

Eiwei\ss{} steif schlagen und Puderzucker unterr\"uhren.
Kalte Pl\"atzchen mit Zuckerguss und Gummib\"archen, Mandeln, N\"ussen verzieren.

Tipp: Das Eiweiss kann durch Kichererbsenwasser ausgetauscht werden.}

