%\rezept{Titel}{20}{Untertitel}{150 & ml & Ei \\3 & Laib & Brot }{Ei aufs Brot spiegeln und mit Seffer und Pfalz bestreuen.}

\rezept{
Chocolate-Chip-Cookies}{20}{
24 Kekse, jeder 160 cals}{
1/2 cup & 115 g & butter, handwarm \\
1/2 cup & 100 g & brauner Zucker \\
1/2 cup & 100 g & Zucker \\
1 & & großes Ei, handwarm \\
1 & & TL vanilla extract \\
1 ¼ cup & 185 g & Mehl \\
1 Tbsp & 7.5 g & Stärke \\
1/2 tsp & 2.5 g & Natron \\
1/2 tsp & 2.5 g & Salz \\
1 1/2 cup & 262 g & backstabile Chocolate Chips \\
1 cup & 100 g & Nüsse (Walnuss, Macadamia...)
}{
Die Butter, den Zucker vermischen, das Ei und die Vanille dazugeben und gut verrühren.
In einer separaten Schüssel das gesiebte Mehl mit der Stärke, dem Natron und Salz vermischen und nach und nach zu der Buttermischung geben und verrühren.
Die Chocolate Chips und Nüsse dazugben und vermischen.
Zu kleinen Bällen formen (etwa 2 TL) und für mindestens eine Stunde im Kühlschrank ruhen lassen.
Den Ofen auf 162 Grad Celsisus vorheizen und Kekse mit etwa 4 cm Abstand auf einem Blech etwa 15-18 Minuten backen bis die Ränder golden werden.
Die Kekse auf dem Blech abkühlen lassen (sie sind jetzt sehr weich und härten beim Abkühlen aus).
Notiz: Statt die Bälle zu backen kann man sie auch einfrieren und Tage später backen, dabei sollte beachtet werden dass sie etwa 20 Minuten bei Raumtemperatur antauen sollten bevor sie gebacken werden.
}

\rezept{Streuselkekse}{20}{
f\"ur 1 Blech}{
200 & g & Mehl \\
100 & g & Zucker \\
100 & g & Butter \\
1 & Pck & Vanillezucker \\
1 & Prise & Salz \\
2-3 & EL & Pflaumenmuß
}{Streusel zusammenkneten und am Ende Marmelade hinzuf\"ugen, sodass eine feuchte, klebrige Masse entsteht.
Zu kleinen Kugeln formen und aufs Blech drücken, braucht kein Backpapier.

15-20 Minuten bei 170 Grad backen, bis die Kekse ein wenig gebräunt sind.}

\newpage

\rezept{Zimtsterne}{20}{
vegan}{
300 & g & Puderzucker \\
2 & EL & Zimt \\
8 & EL & Wasser \\
1 & EL & Zitronensaft \\
150 & g & gehackte Mandeln \\
200 & g & gehackte Haseln\"usse \\
1 & EL & Orangenschale
}{
Vorsicht, der Teig klebt.
Etwa 10 Minuten backen.  }

\rezept{Alfajores}{20}{
für etwa 20}{
75 & g & Margarine \\
50 & g & Puderzucker \\
50 & g & Mehl \\
175 & g & Speisest\"arke \\
1/2 & TL & Backpulver \\
1 & & Ei
}{
Margarine mit Puderzucker verrühren.
Das Ei hinzufügen und verrühren.
Mehl, Speisestärke, Backpulver dazu mischen, im Kühlschrank ruhen lassen.

Ausrollen und ausstechen.

10-12 Minuten bei 160-180 Grad backen.

Mit Marmelade/Manjar/Schokolade zu Doppelkeksen zusammenkleben.

Tip: Etwas Mehl kann durch Backakao ersetzt werden.}

\rezept{Lebkuchen}{20}{
f\"ur ein Lebkuchenhaus}{
175 & g & Honig \\
50 & g & Zucker \\
50 & g & Butter \\
300 & g & Mehl \\
1 & TL & Backpulver \\
1 & EL & Kakao \\
1 & TL & Pfefferkuchengew\"urz \\
2 & & Eiweiss \\
450 & g & Puderzucker
}{
Honig, Zucker und Butter zusammen im Topf erhitzen bis der Zucker sich gel\"ost hat.
Aufpassen, dass der Zucker nicht karamellisiert, Masse abk\"uhlen lassen!

Mehl, Backpulver, Kakao und Pfefferkuchengew\"urz mischen und unter die erkaltete Honigmasse r\"uhren.
Teig eine Stunde ruhen lassen.

Formen ausstechen und bei Umluft 180 Grad, 10-15 min backen.

Eiwei\ss{} steif schlagen und Puderzucker unterr\"uhren.
Kalte Pl\"atzchen mit Zuckerguss und Gummib\"archen, Mandeln, N\"ussen verzieren.

Tipp: Das Eiweiss kann durch Kichererbsenwasser ausgetauscht werden.}

