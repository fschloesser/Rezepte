%\rezept{Titel}{20}{Untertitel}{150 & ml & Ei \\3 & Laib & Brot }{Ei aufs Brot spiegeln und mit Seffer und Pfalz bestreuen.}

\rezept{Streusel}{20}{
f\"ur 1 Blech}{
200 & g & Mehl \\
100 & g & Zucker \\
100 & g & Butter \\
1 & Pck & Vanillezucker \\
1 & Prise & Salz
}{Zusammenmischen bis es kr\"umelt und über Obstkuchen geben oder für Plätzchen verwenden.}

\rezept{Hefeteig}{20}{
ein Blech, f\"ur Blechkuchen}{
400 & g & Mehl \\
1 & Pck. & Hefe \\
1 & TL & Zucker \\
1 & Tasse & (warme) Milch \\
& & oder Wasser \\
6 & EL & \"Ol \\
1 & TL & Salz
}{ Alles zusammen mischen und eine halbe Stunde an einem warmen Ort gehen lassen.

Auf dem Blech ausrollen und mit Obst und Streuseln belegen.
25 bis 30 Minuten bei 175 Grad Umluft backen.}

\rezept{Blueberrypie}{20}{
18 cm Springform oder etwas größer}{
& & Füllung: \\
600 & g  & Blaubeeren \\
30  & g  & Maisstärke \\
100 & g  & Zucker \\
1   & EL & Zimt \\
evtl & etwas & Zitrone \\
& & Teig (Mürbeteig):\\
250 & g  & Mehl \\
125 & g  & Margarine \\
3 & EL & Wasser \\
1   &    & Ei \\
1/2 & TL & Salz
}{
M\"urbeteig mit Ei vorbereiten.
F\"ur die Füllung: Alles mischen und beiseite stellen.
Den Teig in drei gleiche Teile teilen, zwei Kreise, die etwa so groß sind wie die Basis der Form, mit dem dritten Teil die Wand der Form abdecken.
Einen Kreis als Boden in die Springform geben und den Streifen an die Ränder drücken, dann mit der Blaubeermischung füllen und mit dem zweiten Kreis abdecken.
In den Deckel Luftlöcher schneiden und optional mit Eigelb-Milch-Mischung bestreichen.
Im Ofen 20 Minuten auf unterer Schiene bei 220 Grad backen,
Auf 180 Grad herunterschalten und 30 bis 40 Minuten zu Ende backen bis die Säfte bubblen und dicker geworden sind.
Tip: Geht auch in Muffinformen (16 ungedeckt, 12 gedeckt).}

\newpage

\rezept{Hefezopf mit Rosinen}{20}{
zu Ostern}{
500 & g & Mehl\\
& & Hefe \\
1 & Pck & Vanillezucker \\
100 & g & Zucker \\
100 & g & Butter \\
1 & Prise & Salz \\
2 & & Eier \\
80 & g & Rosinen \\
125 & g & Milch
}{
Alles zusammen mischen, eine Stunde gehen lassen.
Zopf flechten und mit Milch bestreichen, eine halbe Stunde gehen lassen.
Eine halbe Stunde bei 175 Grad Umluft backen. }

\rezept{Torta Pompadour}{20}{7000 cals}{
360 & g & Butter \\
360 & g & Puderzucker \\
260 & g & Mandel oder Kokos \\
  2 & Tassen & Milch \\
  5 & & Eigelb \\
  3 & Blatt & Gelatine \\
  4 & Päckchen & Vanillezucker \\
    & Pakete & Katzenzungen
}{
Die Butter mit dem Zucker schaumig schlagen.
Crema Pastelera aus Milch, Zucker, Vanille, Gelatine und Kokos machen und dazu geben.
Für die Torte in einer Springform die Kekse auf dem Boden und an den Rändern auslegen.
Abwechselnd Creme und Kekse in Schichten auslegen bis die Form voll ist.
Mit Merengue bedecken und im Ofen mit voller Hitze eine Minute den Eischnee an den Spitzen karamellisieren.}

\rezept{Crema pastelera}{20}{}{
250 & ml & Milch \\
3 & & Eier \\
2 & EL & Butter \\
2 & EL & Maisstärke \\
3 & EL & Zucker \\
& & Vanille \\
}{
Die Milch mit ein wenig des Zuckers erhitzen.
In einer Schüssel Eier, Stärke, Zucker und Vanille vermischen.
Zum temperieren etwa die Hälfte der Milch dazugeben und mit einem Schneebesen schlagen.
Alles zurück zu der restlichen Milch in den Topf geben, aufkochen und durchgängig schlagen bis die Creme andickt.
In einer Schüssel abgedeckt auskühlen und ruhen lassen.}

\rezept{Bisquit}{20}{sehr fluffy, gut zu schneiden}{
5 & & Eier \\
125 & g & Zucker \\
75 & g & Mehl \\
25 & g & Maismehl \\
12 & g & Kakao
}{
Eier und Zucker über einem Wasserbad vermischen bis der Zucker sich aufgelöst hat, das Ei soll nicht stocken.
Vom Wasserbad nehmen und mit der Maschine aufschlagen.
Gesiebte Mehl-Kakao-Mischung vorsichtig unterheben und im Ofen bei 160 Grad etwa eine halbe Stunde backen.

Fingertest: der Bisquit ist fertig wenn der Fingerabdruck verschwindet.}

\rezept{Tres Leches}{20}{}{
1/2 & Tasse & Milch \\
1 & Prise & Salz \\
6 & & Eier \\
1 1/2 & Tassen & Zucker \\
2 & Tasse & Mehl \\
3 & TL & Backpulver \\
 & & Vanilleextrakt \\
1 & Tasse & gezuckerte Kondensmilch \\
1 & Tasse & Kondensmilch \\
1 & Tasse & Milch \\
}{
Form einfetten.
Mehl und Backpulver dreimal sieben.
Das Eiweiß zu Eischnee schlagen und das Salz hinzufügen.
Nach und nach den Zucker dazu geben und dann die Eigelbe eins nach dem anderen dazu geben.
Ein Drittel des Mehls darunterheben und Milch und Vanille dazu geben.
Den Rest des Mehl hinzufügen.
Bei 160 Grad eine halbe Stunde backen bis goldenbraun.
In einer Tasse die drei Milch mischen (zimmerwarm).
Den Bisquit aus dem Ofen nehmen, und die Oberfläche mit einem Zahnstocher einstechen.
Mit der Milchmischung nach und nach den Bisquit durchnässen.
Im Kühlschrank auskühlen lassen.
Nach Geschmack mit Crema Pastelera, Merengue, gerösteten Mandeln dekorieren und servieren.
}
