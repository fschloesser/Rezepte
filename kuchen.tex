%\rezept{Titel}{20}{Untertitel}{150 & ml & Ei \\3 & Laib & Brot }{Ei aufs Brot spiegeln und mit Seffer und Pfalz bestreuen.}

\rezept{Streusel}{20}{
f\"ur 1 Blech}{
200 & g & Mehl \\
100 & g & Zucker \\
100 & g & Butter \\
1 & Pck & Vanillezucker \\
1 & Prise & Salz
}{Zusammenmischen bis es kr\"umelt und über Obstkuchen geben oder für Plätzchen verwenden.}

\rezept{Hefeteig}{20}{
ein Blech, f\"ur Blechkuchen}{
400 & g & Mehl \\
1 & Pck. & Hefe \\
1 & TL & Zucker \\
1 & Tasse & (warme) Milch \\
& & oder Wasser \\
6 & EL & \"Ol \\
1 & TL & Salz
}{ Alles zusammen mischen und eine halbe Stunde an einem warmen Ort gehen lassen.

Auf dem Blech ausrollen und mit Obst und Streuseln belegen.
25 bis 30 Minuten bei 175 Grad Umluft backen.}

\rezept{Blueberrypie}{20}{
18 cm Springform oder etwas größer}{
& & Füllung: \\
600 & g  & Blaubeeren \\
30  & g  & Maisstärke \\
100 & g  & Zucker \\
1   & EL & Zimt \\
evtl & etwas & Zitrone \\
& & Teig (Mürbeteig):\\
250 & g  & Mehl \\
125 & g  & Margarine \\
3 & EL & Wasser \\
1   &    & Ei \\
1/2 & TL & Salz
}{
M\"urbeteig mit Ei vorbereiten.

F\"ur die Füllung: Alles mischen und beiseite stellen.

Den Teig in drei gleiche Teile teilen, zwei Kreise, die etwa so groß sind wie die Basis der Form, mit dem dritten Teil die Wand der Form abdecken.
Einen Kreis als Boden in die Springform geben und den Streifen an die Ränder drücken, dann mit der Blaubeermischung füllen und mit dem zweiten Kreis abdecken.
In den Deckel Luftlöcher schneiden und optional mit Eigelb-Milch-Mischung bestreichen.

Im Ofen 20 Minuten auf unterer Schiene bei 220 Grad backen,
Auf 180 Grad herunterschalten und 30 bis 40 Minuten zu Ende backen bis die Säfte bubblen und dicker geworden sind.

Tip: Geht auch in Muffinformen (16 ungedeckt, 12 gedeckt).}

\newpage

\rezept{Hefezopf mit Rosinen}{20}{
zu Ostern}{
500 & g & Mehl\\
& & Hefe \\
1 & Pck & Vanillezucker \\
100 & g & Zucker \\
100 & g & Butter \\
1 & Prise & Salz \\
2 & & Eier \\
80 & g & Rosinen \\
125 & g & Milch
}{
Alles zusammen mischen, eine Stunde gehen lassen.

Zopf flechten und mit Milch bestreichen, eine halbe Stunde gehen lassen.

Eine halbe Stunde bei 175 Grad Umluft backen. }

