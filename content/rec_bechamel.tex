\rezept{Bechamelso\ss{}e}{20}{800g; reicht knapp f\"ur eine Linsenlasagne}{
50 & g & Butter \\
50 & g & Mehl \\
600 & g & Gem\"usebr\"uhe \\
100 & g & Schlagsahne \\
50 & g & frisch geriebener Parmesan \\
 & & frisch geriebene Muskatnuss
}{
\label{rec:bechamel}
Die Butter in einem Topf zerlassen und das Mehl darin unter R\"uhren mit einem Holzl\"offel and\"unsten, bis es Blasen wirft und Butter und Mehl sich gut miteinander verbunden haben.
Daf\"ur zun\"achst einen Holzl\"offel nehmen, damit die Mehl-Butter-Mischung nicht im Schneebesen h\"angen bleibt.

Br\"uhe (Zimmertemperatur) nach und nach dazugießen und dabei kr\"aftig mit einem Holzl\"offel r\"uhren, damit keine Kl\"umpchen entstehen.
Unter st\"andigem R\"uhren bei kleiner Hitze etwa 3 Minuten kochen.
Daf\"ur nun den Schneebesen nehmen, damit die So\ss{}e garantiert glatt und sch\"on cremig wird.

Hitze ausschalten, Sahne und den Parmesan unterr\"uhren und die So\ss{}e mit Salz, Pfeffer und frisch geriebenem Muskat abschmecken (Muskat sparsam dosieren!).
Topf mit der So\ss{}e vom Herd nehmen.  }
