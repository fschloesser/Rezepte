\rezept{
Banoffee Pie
}{20}{
5662 kcal; Kuchen von etwa 23 cm Durchmesser
}{
Boden & & \\
100 & g & Haferflocken \\
150 & g & Weizenvollkornmehl \\
 50 & g & brauner Zucker \\
2.5 & g & Natron \\
2.5 & g & Salz \\
175 & g & kalte Butter \\
  1 &   & großes Ei \\
& & \\
Füllung & & \\
  3 &    & Bananen \\
115 &  g & Butter \\
100 &  g & brauner Zucker \\
300 & ml & gezuckerte Kondensmilch \\
  5 & ml & Vanilleextrakt \\
2.5 &  g & Salz \\
& & \\
Topping & & \\
200 &  g & Schlagsahne \\
 16 &  g & Puderzucker \\
  2 & ml & Vanilleextrakt \\
}{
1.
Für den Teig die Haferflocken, den Zucker, das Natron und Salz in einem Mixer vermixen bis die Haferflocken kleingehackt sind.
Die Butter dazu geben und mixen bis der Teig einer krümelige Textur hat.
Das Ei hinzugeben und gründlich mischen bis der Teig zusammenkommt.
Eine Kuchenbodenform damit auslegen und mit einer Gabel den Boden einstechen.
Im Kühlschrank für mindestens eine Stunde kalt stellen.

2.
Den Ofen auf 160 Grad Celsius vorheizen und dann den Boden 20-25 Minuten backen bis einheitlich golden.
Mit dem Füllen warten bis der Boden abgekühlt ist.

3.
Für die Füllung die Bananen in dicke Scheiben schneiden und damit den Boden gründlich auslegen.
Die Butter mit dem Zucker in einem Topf über mittelhoher Hitze zum blubbern bringen und dann die gezuckerte Kondensmilch hinzugeben.
Unter ständigem Rühren zum Kochen bringen und dann eine Minute kochen.
Von der Hitze nehmen und Vanille und Salz darunter mischen.
Das Toffee über die Bananen gießen und sicherstellen dass die Bananen bedeckt sind.
Auf Raumtemperatur abkühlen lassen und mindestens eine Stunde im Kühlschrank ruhen lassen.

4.
Die Sahne schlagen und über den Kuchen geben.
Zum Schluss nach Belieben mit gerösteten Mandelspalten bestreuen.
Bis zum Servieren im Kühlschrank aufbewahren.
}
