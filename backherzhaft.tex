%\rezept{Titel}{20}{Untertitel}{150 & ml & Ei \\3 & Laib & Brot }{Ei aufs Brot spiegeln und mit Seffer und Pfalz bestreuen.}

\rezept{Marraqueta}{20}{tradicional}{
1 & kg & Mehl \\
650 & ml & lauwarmes Wasser \\
1 & EL & Salz \\
1 & TL & Zucker \\
10 & g & Trockenhefe
}{
Die Hefe mit dem Zucker und 4 EL des Wassers auflösen und etwa 5 Minuten stehen lassen.
Das Salz und die Hefe mit dem Mehl in eine Schüssel geben und mit dem restlichen Wasser verrühren.
Den Teig 10 Minuten lang kneten bis er nicht mehr klebt und in einer Schüssel etwa 30 Minuten an einem warmen Ort gehen lassen bis er sich auf die doppelte Menge vermehrt hat.
Die Luft herauskneten und in 12 kleine Kugeln formen, von denen jeweils zwei nebeneinander aufs gemehlte Blech gelegt werden.
Mit einem Holzlöffel kräftig eine Linie in die Mitte drücken und nochmals 10 Minuten gehen lassen bedeckt mit einem feuchten Tuch.
Im vorgeheizten Ofen 15 bis 20 Minuten backen, dabei eine Schüssel Wasser in den Ofen stellen um die Luft feucht zu halten.
}

\rezept{Els\"asser Flammkuchen}{20}{
Einfach und gut - f\"ur 4 Portionen}{
440 & g & Mehl (550) \\
200 & ml & lauwarmes Wasser \\
6 & EL & \"Ol \\
2 & & Eigelb \\
1  & TL & Salz \\
500 & g & Zwiebeln \\
1 & Becher & saure Sahne \\
1 & Becher & s\"u\ss{}e Sahne
}{
Das Mehl in eine Sch\"ussel sieben, Wasser, \"Ol, Eigelb und Salz zugeben, zu einem glatten Teig verkneten und zugedeckt 30 Minuten an einem zimmerwarmen Ort ruhen lassen.

Zwiebeln halbieren, in feine Scheiben hobeln.
S\"u\ss{}e und saure Sahne mit etwas Salz und Pfeffer verquirlen.

Teig vierteln.
Jedes St\"uck auf einem gro\ss{}en St\"uck Backpapier zu einer hauchd\"unnen rechteckigen Platte auswellen.
Sahnemischung darauf verstreichen und mit Zwiebeln bestreuen.

Bei Umluft 220 Grad etwa 10-15 Minuten goldbraun backen.  }

\rezept{Pizza-\"Ol-Teig}{20}{
Einfach}{
500 & g & Mehl\\
1 & TL & Salz \\
6 & EL & \"Ol \\
1 & Pck. & Hefe \\
300 & ml & Wasser
}{
Mehl, Salz und \"Ol vermischen.
1 W\"urfel Hefe in die Mitte br\"oseln.
Wasser dazugeben und durchkneten bis der Teig sich vom Rand l\"ost.

Tuch dr\"uberlegen und ruhen lassen.  }

\newpage

\rezept{M\"urbeteig}{20}{
für 16 Muffins, 6 kleine Tartes oder eine große}{
200 & g & Mehl \\
100 & g & Butter/Margarine \\
1/2 & TL & Salz \\
3 & EL & Wasser \\
1 & optionales & Ei
}{Teig kneten und in eine gefettete Backform geben.
Das Ei ist optional, ohne ist aber unerprobt.

Vorsicht: die Butter nicht weich werden lassen und den Teig nicht zu lang kneten.
Er ist fertig und muss in den Kühlschrank wenn er eine Masse ist aber man Butter und Mehl noch unterscheiden kann.

Tip: Für Muffinförmchen: für Boden Kreise ausstechen und separat auslegen, für den Rand einen Streifen schneiden und ankleben.

Mal probieren: süßer Mürbeteig (1,2,3-Teig), 1 Teil Zucker, 2 Teile Butter, 3 Teile Mehl.
}

\rezept{Quiche}{20}{
für 16 Muffins, 6 kleine Tartes oder eine große}{
2-3 & & Eier \\
2 & & Tomaten \\
1 & & Zwiebel \\
6 & Zehen & Knoblauch \\
150 & ml & Milch \\
100 & g & Schafskäse \\
1 & & Zucchini \\
6 & & Karotten
}{
Mürbeteig mit Ei vorbereiten und kalt stellen wie oben beschrieben.
Gemüse reiben und mit Zwiebeln, Knoblauch in Olivenöl andünsten, Form fetten.
Eier und Milch mit Salz, Pfeffer und Kräutern vermischen.
Teig dünn ausrollen und in die Form drücken, mit dem Gemüse belegen und die Eiermischung darüber gießen.
Den Schafskäse darüber krümeln.
Bei 200 Grad etwa 30 bis 40 Minuten backen.
}

