\rezept{Sauerteigbrot}{20}{Tagesprojekt, 2 Brote}{
Levain & & (100\% Hydration) \\
35 & g & Starter \\
35 & g & Weizenvollkornmehl \\
35 & g & Weizenmehl \\
70 & g & Wasser, Raumtemperatur \\
Teig & & (80\% Hydration) \\
175 & g & Levain \\
800 & g & Mehl (Mischung) \\
75 & g & Vollkornmehl \\
700 & g & Wasser \\
20 & g & Salz
}{
Brot:

 8:00 am: Levain ansetzen, dafür alle Zutaten mischen und 5-6 Stunden bei etwa 25 C mit einem feuchten Handtuch abgedeckt ruhen lassen.

11:30 am: Teig ansetzen, dafür Mehl und 620g Wasser mischen und dann mit einem feuchten Handtuch abgedeckt ruhen lassen.

 1:00 pm: Wenn der Levain fertig ist (gerade dann wenn die Oberfläche anfängt zu fallen) zum Teig geben, hineindrücken und vermischen. Slap-and-fold ohne Mehl bis der Teig nicht mehr so schlimm klebt, dann mit einem feuchten Handtuch abgedeckt 25 min ruhen lassen.
Sauerteig nie kneten, nur falten und schlagen.

 1:30 pm: Mit dem Salz und dem restlichen Wasser mischen und slap-and-fold ohne Mehl.

 1:35 pm: Gehen/Fermentieren lassen bei 25 Celsius.

 1:50 pm: Den Teig in der Schüssel falten: den Teig vom Rand hochziehen und über sich selbst falten, einmal um die Schüssel herum. Nr. 1

 2:05 pm: Falten Nr. 2

 2:20 pm: Falten Nr. 3

 2:50 pm: Falten Nr. 4

 3:20 pm: Falten Nr. 5

 3:50 pm: Falten Nr. 6 und weiterhin ruhen lassen.

 6:00 pm: Auf die ungemehlte Arbeitsfläche geben und mit angefeuchteten Händen und angefeuchtetem Teigschaber In zwei teilen und ohne zu kneten grob rund formen.

 6:20 pm: Unbedeckt aber oberflächlich gemehlt auf der Arbeitsfläche 20 Minuten ruhen lassen. Umdrehen und zusammenfalten und in mit großzügig gemehlten Handtüchern ausgelegten Schalen ruhen lassen mit dem Bauch nach oben. (Im Kühlschrank 12-14 Stunden wenn nötig über Nacht.)

 8:20 am oder am nächsten Morgen: Den ersten Laib backen (Poke test: Wenn der Teig fast gänzlich aber nicht komplett zurück springt hat er fertig geruht.). Dutch Oven bei 260 Celsius vorheizen, Brot mehlen und Dutch Oven mehlen, Laib hineinkippen und leicht asymmetrisch einen halben cm tief einschneiden. Erst 20 Minuten lang bei 260 Celsius dann 20-30 Minuten 230 Celsius ohne Deckel backen Herausholen wenn das Brot eine gute Farbe hat. Danach den zweiten Laib genauso backen, dafür den Topf 15 Minuten aufheizen lassen. (Dutch oven - combo cooker - kombi-topf: Eine Pfanne
 und ein Topf, gusseisern die bei Bedarf zusammenpassen.)
Für die zweiten zwanzig Minuten ein Blech als Hitzeschirm unter den Topf schieben.

Vorm Anschneiden eine Stunde auf einem Rost oder im halboffenen Ofen auskühlen lassen.

Starter:

In einem Halbliterglas mit aufgelegtem Deckel an einem dunklen, etwa 25 Grad warmen Ort lagern.
Das Gewicht des Glases vorher notieren und Mengen mit einer Waage ausmessen.

Tag 1:
150g handwarmes Wasser und
100g Bio-Roggenvollkornmehl mischen bis die Masse keine Klumpen mehr hat. Mit einem lose sitzenden Deckel abdecken und 12 bis 24 Stunden ruhen lassen.

Tag 2+3:
70g Starter,
50g Bio-Roggenvollkornmehl,
50g ungebleichtes Weizenmehl und
115g handwarmes Wasser

Tag 4+5:
70g Starter,
50g Bio-Roggenvollkornmehl,
50g ungebleichtes Weizenmehl und
100g handwarmes Wasser

Tag 6:
50g Starter,
50g Bio-Roggenvollkornmehl,
50g ungebleichtes Weizenmehl und
100g handwarmes Wasser

Tag 7:
25g Starter,
50g Bio-Roggenvollkornmehl,
50g ungebleichtes Weizenmehl und
100g handwarmes Wasser

Tipp: Nicht mehr benötigten Starter in einer geölten Pfanne zu einem Pfannkuchen backen, z.B. mit Sesam bestreuen.
}

\rezept{Hefebrot}{20}{dauert mind. 4 Stunden}{
500 & g & Mehl \\
1 & TL & Salz \\
10 & g & Trockenhefe \\
320 & g & Wasser \\
& & optional Zwiebel, Kerne, \\
& & getrocknete Tomate, \\
& & Rosinen etc.
}{
Das Mehl, Salz und die Hefe mischen, mit dem Wasser vermengen.
Es mag sein, dass die Mischung zu trocken oder zu flüssig erscheint, dies sollte nicht der Fall sein.
Auf einer gemehleten Oberfläche 10 bis 15 Minuten zu einem elastischen Teig verkneten der nicht mehr an der Arbeitsfläche klebt.
Mit einer dünnen Schicht Öl bedecken und an einem warmen Ort (unter 55 Grad, zum Beispiel auf einer handwarmen Heizung) in einer mit einem Küchentuch bedecketen Schüssel 1 bis 2 Stunden (oder bis sich das Volumen verdoppelt hat) ruhen lassen.
Aus der Schüssel nehmen und nochmal 1 bis 2 Minuten kneten, dabei die handtuchtrockenen Extras hinzufügen.
In eine geölte Backform geben oder in der gewünschten Form auf ein Blech legen und nochmal an einem warmen Ort 1/2 bis 1 Stunde (oder bis sich das Volumen verdoppelt hat) gehen lassen.
Im 230 Grad Celsius vorgeheizten Ofen etwa 20 bis 40 Minuten backen (je nach Form).
Wenn die Oberfläche golden wird das Brot aus dem Ofen nehmen und vor dem Anschneiden auf dem Rost 10 bis 15 Minuten auskühlen lassen.

Notiz: Die wichtigen zwei Dinge sind, dass der Teig gut geknetet wird um die Glutenstränge zu entwickeln und dass er lange genug ruht, denn in dieser Phase entwickelt sich der Geschmack. }

\rezept{Marraqueta}{20}{tradicional}{
1 & kg & Mehl \\
650 & ml & lauwarmes Wasser \\
1 & EL & Salz \\
1 & TL & Zucker \\
10 & g & Trockenhefe
}{
Die Hefe mit dem Zucker und 4 EL des Wassers auflösen und etwa 5 Minuten stehen lassen.
Das Salz und die Hefe mit dem Mehl in eine Schüssel geben und mit dem restlichen Wasser verrühren.
Den Teig 10 Minuten lang kneten bis er nicht mehr klebt und in einer Schüssel etwa 30 Minuten an einem warmen Ort gehen lassen bis er sich auf die doppelte Menge vermehrt hat.
Die Luft herauskneten und in 12 kleine Kugeln formen, von denen jeweils zwei nebeneinander aufs gemehlte Blech gelegt werden.
Mit einem Holzlöffel kräftig eine Linie in die Mitte drücken und nochmals 10 Minuten gehen lassen bedeckt mit einem feuchten Tuch.
Im vorgeheizten Ofen 15 bis 20 Minuten backen, dabei eine Schüssel Wasser in den Ofen stellen um die Luft feucht zu halten.
}

\rezept{Pizza-\"Ol-Teig}{20}{
Einfach}{
500 & g & Mehl\\
1 & TL & Salz \\
6 & EL & \"Ol \\
1 & Pck. & Hefe \\
300 & ml & Wasser
}{
Mehl, Salz und \"Ol vermischen.
1 W\"urfel Hefe in die Mitte br\"oseln.
Wasser dazugeben und durchkneten bis der Teig sich vom Rand l\"ost.

Tuch dr\"uberlegen und ruhen lassen.  }

\newpage

\rezept{Els\"asser Flammkuchen}{20}{
Einfach und gut - f\"ur 4 Portionen}{
440 & g & Mehl (550) \\
200 & ml & lauwarmes Wasser \\
6 & EL & \"Ol \\
2 & & Eigelb \\
1  & TL & Salz \\
500 & g & Zwiebeln \\
1 & Becher & saure Sahne \\
1 & Becher & s\"u\ss{}e Sahne
}{
Das Mehl in eine Sch\"ussel sieben, Wasser, \"Ol, Eigelb und Salz zugeben, zu einem glatten Teig verkneten und zugedeckt 30 Minuten an einem zimmerwarmen Ort ruhen lassen.

Zwiebeln halbieren, in feine Scheiben hobeln.
S\"u\ss{}e und saure Sahne mit etwas Salz und Pfeffer verquirlen.

Teig vierteln.
Jedes St\"uck auf einem gro\ss{}en St\"uck Backpapier zu einer hauchd\"unnen rechteckigen Platte auswellen.
Sahnemischung darauf verstreichen und mit Zwiebeln bestreuen.

Bei Umluft 220 Grad etwa 10-15 Minuten goldbraun backen.  }

\rezept{M\"urbeteig}{20}{
für 16 Muffins, 6 kleine Tartes oder eine große}{
200 & g & Mehl \\
100 & g & Butter/Margarine \\
1/2 & TL & Salz \\
3 & EL & Wasser \\
1 & optionales & Ei
}{Teig kneten und in eine gefettete Backform geben.
Das Ei ist optional, ohne ist aber unerprobt.

Vorsicht: die Butter nicht weich werden lassen und den Teig nicht zu lang kneten.
Er ist fertig und muss in den Kühlschrank wenn er eine Masse ist aber man Butter und Mehl noch unterscheiden kann.

Tip: Für Muffinförmchen: für Boden Kreise ausstechen und separat auslegen, für den Rand einen Streifen schneiden und ankleben.

Mal probieren: süßer Mürbeteig (1,2,3-Teig), 1 Teil Zucker, 2 Teile Butter, 3 Teile Mehl.
}

\rezept{Quiche}{20}{
für 16 Muffins, 6 kleine Tartes oder eine große}{
2-3 & & Eier \\
2 & & Tomaten \\
1 & & Zwiebel \\
6 & Zehen & Knoblauch \\
150 & ml & Milch \\
100 & g & Schafskäse \\
1 & & Zucchini \\
6 & & Karotten
}{
Mürbeteig mit Ei vorbereiten und kalt stellen wie oben beschrieben.
Gemüse reiben und mit Zwiebeln, Knoblauch in Olivenöl andünsten, Form fetten.
Eier und Milch mit Salz, Pfeffer und Kräutern vermischen.
Teig dünn ausrollen und in die Form drücken, mit dem Gemüse belegen und die Eiermischung darüber gießen.
Den Schafskäse darüber krümeln.
Bei 200 Grad etwa 30 bis 40 Minuten backen.
}
