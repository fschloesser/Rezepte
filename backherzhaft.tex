%\rezept{Titel}{20}{Untertitel}{150 & ml & Ei \\3 & Laib & Brot }{Ei aufs Brot spiegeln und mit Seffer und Pfalz bestreuen.}

\rezept{Marraqueta}{20}{tradicional}{
1 & kg & Mehl \\
650 & ml & lauwarmes Wasser \\
1 & EL & Salz \\
1 & TL & Zucker \\
10 & g & Trockenhefe
}{
Die Hefe mit dem Zucker und 4 EL des Wassers auflösen und etwa 5 Min stehen lassen.
Das Salz und die Hefe mit dem Mehl in eine Schüssel geben und mit dem restlichen Wasser verrühren.
Den Teig 10 Minuten lang kneten bis er nicht mehr klebt und in einer Schüssel etwa 30 Minuten an einem warmen Ort gehen lassen bis er sich auf die doppelte Menge vermehrt hat.
Die Luft herauskneten und in 12 kleine Kugeln formen, von denen jeweils zwei nebeneinander aufs gemehlte Blech gelegt werden.
Mit einem Holzlöffel kräftig eine Linie in die Mitte drücken und nochmals 10 Minuten gehen lassen bedeckt mit einem feuchten Tuch.
Im vorgeheizten Ofen 15 bis 20 Minuten backen, dabei eine Schüssel Wasser in den Ofen stellen um die Luft feucht zu halten.
}

\rezept{Flammkuchen}{20}{
}{
500 & g & Mehl \\
250 & g & Wasser \\
2 & Prisen & Salz \\
4 & EL & \"Ol \\
& & \\
4 & & ged\"unstete Zwiebeln \\
2 & Becher & saure Sahne \\
1 & Becker & Quark \\
& & Pfeffer \\
& & Salz \\
& & Schnittlauch
}{Teig anr\"uhren und d\"unn ausrollen.
Mit Quark und Sahne bestreichen, Zwiebeln dr\"uber streuen und mit Pfeffer und Salz w\"urzen.
Im Backofen auf Backpapier backen.

Zum Schluss mit Schnittlauch bestreuen.}

\newpage

\rezept{Pizza-\"Ol-Teig}{20}{
Einfach}{
500 & g & Mehl\\
1 & TL & Salz \\
6 & EL & \"Ol \\
1 & Pck. & Hefe \\
300 & ml & Wasser
}{
Mehl, Salz und \"Ol vermischen.
1 W\"urfel Hefe in die Mitte br\"oseln.
Wasser dazugeben und durchkneten bis der Teig sich vom Rand l\"ost.

Tuch dr\"uberlegen und ruhen lassen.  }

\newpage

\rezept{Quarkbrot}{20}{
von Silvana}{
150 & g & Quark \\
500 & g & Mehl \\
2 & TL & Zucker \\
2 & TL & Salz \\
2 & Pck. & Backpulver \\
2 & & Eier
}{
Bei 180 Grad Ober- und Unterhitze ca. 45 min backen.  }

\rezept{M\"urbeteig}{20}{
f\"ur Tartes, Quiche}{
200 & g & Mehl \\
100 & g & Butter/Margarine \\
1/2 & TL & Salz \\
3 & EL & Wasser \\
1 & optionales & Ei
}{Teig kneten und in eine gefettete Backform geben.
Das Ei ist optional, ohne ist aber unerprobt.
VORSICHT: die butter nicht weich werden lassen und den Teig nicht zu lang kneten.
Er ist fertig und muss in den Kühlschrank wenn er eine Masse ist aber man Butter und Mehl noch unterscheiden kann.
}

\rezept{Quiche}{20}{
für 6 kleine Tartes oder eine große}{
 & & Teig: \\
200 & g & Mehl \\
100 & g & Butter/Margarine \\
1/2 & TL & Salz \\
3 & EL & Wasser \\
1 & & Ei \\
 & & Füllung: \\
2 & & Eier \\
2 & & Tomaten \\
1 & & Zwiebel \\
6 & Zehen & Knoblauch \\
150 & ml & Milch \\
100 & g & Schafskäse
}{
Mürbeteig vorbereiten und kalt stellen wie oben beschrieben.
Gemüse schneiden, Zwiebeln, Knoblauch in Olivenöl andünsten, Form fetten.
Eier und Milch mit Salz, Pfeffer und Kräutern vermischen.
Teig dünn ausrollen und in die Form drücken, mit dem Gemüse belegen und die Eiermischung darüber gießen.
Den Schafskäse darüber krümeln.
Bei 200 Grad etwa 30 bis 40 Minuten backen.
}

