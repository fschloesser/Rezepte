\tips{Küche}{
\item Messer immer scharf halten, nicht über das Brettchen ziehen um geschnittenes Gemüse herunter zu schieben.
\item Ein Handtuch in Griffweite halten.
\item Nicht Reiben (microplane zester), Mandolinen, Juliennes vergessen (dafür gibt es auch Schutzhandschuhe)
\item Siebe und Chinoissiebe benutzen um Soßen zu passieren.
\item Utensilien im Restauranthandel, nicht im Küchengeschäft kaufen.
\item Restaurantgeheimnisse sind: Butter, Schalotten, Brühe, Salz und Säure
\item Vor dem Kochen Zutaten waschen, schneiden, vorbereiten und abwiegen (mise-en-place).
\item Ein Brettchen oder eine Schale auf ein nasses Handtuch stellen um es an Ort und Stelle zu halten.
\item Eine Kompost-Schüssel ausgelegt mit einer alten Zeitung benutzen anstatt einer Mülltüte.
\item Die Küche beim Arbeiten sauber machen und sauber halten, immer genug Arbeitsplatz sauber und frei haben.
}

\tips{Kochen}{
\item Die Pfanne nicht zu voll machen.
\item Säure hinzufügen gegen Salz (zum Beispiel Essig, Zitronensaft oder Wein)
\item Soßen andicken mit Stärke: 1 TL Stärke mit 1 TL Wasser vermischen und hinzufügen.
\item Frische Gewürze in Butter oder Öl einfrieren.
\item Soße oder Kompott zu sauer? Etwas Natron (1/4 TL) hinzufügen, aber nicht zu viel!
\item Gewürze oder Nüsse anrösten um den Geschmack zu verbessern.
\item Auf kleiner Hitze kochen: Gemüse fällt nicht so schnell auseinander und Fleisch wird nicht trocken und zäh.
\item Großzügig sein mit Salz.
\item Immer etwas Säure zum Essen hinzufügen.
\item Pfannenreste ablöschen und als Soßenbasis benutzen.
\item Nudeln in der Soße zu Ende kochen. Falls die Soße zu dick wird, mit etwas aufbewahrtem Nudelwasser verlängern.
\item Geriebenen Käse oder Käsewürfel einfrieren und zum überbacken benutzen.
\item Gemüse frittieren um es zu karamellisieren und den Geschmack zu verbessern.
\item Beschichtete Pfannen nur für Eier, Pfannkuchen und French-toast benutzen. (Beschichtung hält Hitze nicht gut aus)
}

\tips{Backen}{
\item Äpfel zum Backen: Jonagold, Elstar, Boskoop, Braeburn (hoher Säuregehalt, fest, nicht mehlig)
\item Einfetten nur mit Butter oder Margarine. Öl hinterlässt schwer zu reinigenden Film (verharzt).
\item Gefrorene Butter mit einer Reibe benutzen.
\item Lange geknetete Teile kleben, weil der Kleber (das Gluten) im Mehl aktiviert wird.
\item Keksteig vor dem Backen in den Kühlschrank stellen und im vorgeheizten Ofen backen, dann zerlaufen sie nicht.
\item Brotteig lang und gut kneten um eine gute Struktur zu erhalten. Keks- und Kuchenteig so wenig wie möglich kneten.
\item Brotteig lange genug ruhen lassen, in dieser Zeit entwickeln sich Geschmäcker und Aromen.
}

\newpage

\tips{Lagerung}{
\item Soße in Eiswürfelformen einfrieren.
\item Tomatensoße zu sauer? Länger kochen oder Zucker hinzufügen.
\item Avocados in einer Papiertüte mit einer Banane reifen schneller, im Kühlschrank reifen sie langsamer.
\item Brot, Fleisch oder Lasagne: Nach dem Kochen vor dem Anschneiden ruhen lassen.
\item Gemüse und frische Kräuter in einer Dose frisch halten mit einem Handtuch zusammen, um überschüssige Feuchtigkeit aufzunehmen.
\item Ausgekratzte Vanilleschoten in einem Glas mit Zucker aufbewahren um Vanillezucker zu machen.
\item Gemüsereste, Knoblauch- und Zwiebelschalen, Knochen aufbewahren und einfrieren um eigene Brühe zu machen.
}

\tips{Fermentation}{
\item Beste Temperatur: 20 Grad. Im Kühlschrank verlangsamt die Fermentation und die Haltbarkeit verlängert sich auf mehrerer Monate.
\item Wasser und Seife genügen, es braucht nicht alles steril zu sein.
\item Während der Fermentation kann eine weiße Hefeschicht auf der Oberfläche entstehen, das ist normal und nicht schädlich und kann einfach entfernt werden.
\item Die Gefäße nicht luftdicht abschließen, Luft muss entweichen können.
\item Luftdicht abgeschlossene Fermentationen entwickeln eine Art Sprudel.
\item Das Fermentat muss komplett unter Wasser liegen.
\item Die Salzlake sollte etwa 2 Prozent betragen.
\item Fermentationszeit ist zwischen 2 Tagen und 2 Wochen.
\item Süße Dinge fermentieren schneller.
\item Fermentation ist fertig, wenn es nicht weiter blubbert und schäumt.
\item Es gibt zwei Gärphasen: eine sehr aktive und eine weniger aktive.
}

