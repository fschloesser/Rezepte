%\rezept{Titel}{20}{Untertitel}{150 & ml & Ei \\3 & Laib & Brot }{Ei aufs Brot spiegeln und mit Seffer und Pfalz bestreuen.}

\rezept{Marraqueta}{20}{tradicional}{
1 & kg & Mehl \\
650 & ml & lauwarmes Wasser \\
1 & EL & Salz \\
1 & TL & Zucker \\
10 & g & Trockenhefe
}{
Die Hefe mit dem Zucker und 4 EL des Wassers auflösen und etwa 5 Min stehen lassen.
Das Salz und die Hefe mit dem Mehl in eine Schüssel geben und mit dem restlichen Wasser verrühren.
Den Teig 10 Minuten lang kneten bis er nicht mehr klebt und in einer Schüssel etwa 30 Minuten an einem warmen Ort gehen lassen bis er sich auf die doppelte Menge vermehrt hat.
Die Luft herauskneten und in 12 kleine Kugeln formen, von denen jeweils zwei nebeneinander aufs gemehlte Blech gelegt werden.
Mit einem Holzlöffel kräftig eine Linie in die Mitte drücken und nochmals 10 Minuten gehen lassen bedeckt mit einem feuchten Tuch.
Im vorgeheizten Ofen 15 bis 20 Minuten backen, dabei eine Schüssel Wasser in den Ofen stellen um die Luft feucht zu halten.
}

\rezept{Dinkelpl\"atzchen}{20}{
simpel}{
1 & Bio & Orange \\
175 & g & Puderzucker \\
200 & g & Butter \\
1 & & Eigelb \\
1 1/2 & EL & Milch \\
400 & g & Dinkelmehl
}{
Teig vor dem Ausstechen eine Stunde im K\"uhlschrank stehen lassen.
Etwa 5 Min backen.  }

\rezept{Flammkuchen}{20}{
}{
500 & g & Mehl \\
250 & g & Wasser \\
2 & Prisen & Salz \\
4 & EL & \"Ol \\
& & \\
4 & & ged\"unstete Zwiebeln \\
2 & Becher & saure Sahne \\
1 & Becker & Quark \\
& & Pfeffer \\
& & Salz \\
& & Schnittlauch
}{Teig anr\"uhren und d\"unn ausrollen.
Mit Quark und Sahne bestreichen, Zwiebeln dr\"uber streuen und mit Pfeffer und Salz w\"urzen.
Im Backofen auf Backpapier backen.

Zum Schluss mit Schnittlauch bestreuen.}

\rezept{Hefeteig}{20}{
ein Blech, f\"ur Blechkuchen}{
400 & g & Mehl \\
1 & Pck. & Hefe \\
1 & TL & Zucker \\
1 & Tasse & Milch \\
& & oder Wasser \\
6 & EL & \"Ol \\
1 & TL & Salz
}{ Alles zusammen mischen und eine halbe Stunde an einem warmen Ort gehen lassen.

Auf dem Blech ausrollen und mit Obst und Streuseln belegen.
25 bis 30 Minuten bei 175 Grad Umluft backen.}

\newpage

\rezept{Hefezopf mit Rosinen}{20}{
zu Ostern}{
500 & g & Mehl\\
& & Hefe \\
1 & Pck & Vanillezucker \\
100 & g & Zucker \\
100 & g & Butter \\
1 & Prise & Salz \\
2 & & Eier \\
80 & g & Rosinen \\
125 & g & Milch
}{
Alles zusammen mischen, eine Stunde gehen lassen.

Zopf flechten und mit Milch bestreichen, eine halbe Stunde gehen lassen.

Eine halbe Stunde bei 175 Grad Umluft backen. }

\rezept{Lebkuchen}{20}{
f\"ur ein Lebkuchenhaus}{
175 & g & Honig \\
50 & g & Zucker \\
50 & g & Butter \\
300 & g & Mehl \\
1 & TL & Backpulver \\
1 & EL & Kakao \\
1 & TL & Pfefferkuchengew\"urz \\
2 & & Eiweiss \\
450 & g & Puderzucker
}{
Honig, Zucker und Butter zusammen im Topf erhitzen bis der Zucker sich gel\"ost hat.
Aufpassen, dass der Zucker nicht karamellisiert, Masse abk\"uhlen lassen!

Mehl, Backpulver, Kakao und Pfefferkuchengew\"urz mischen und unter die erkaltete Honigmasse r\"uhren.
Teig eine Stunde ruhen lassen.

Formen ausstechen und bei Umluft 180 Grad, 10-15 min backen.

Eiwei\ss{} steif schlagen und Puderzucker unterr\"uhren.
Kalte Pl\"atzchen mit Zuckerguss und Gummib\"archen, Mandeln, N\"ussen verzieren.

Tipp: Das Eiweiss kann durch Kichererbsenwasser ausgetauscht werden.}

\newpage

\rezept{M\"urbeteig}{20}{
f\"ur Tartes, Quiche}{
200 & g & Mehl \\
100 & g & Butter \\
1/2 & TL & Salz \\
3 & EL & Wasser \\
(1 & & Ei)
}{Teig kneten und in eine gefettete Backform geben.}

\rezept{Pfannkuchen}{20}{
}{
3 & & Eier \\
200 & g & Mehl \\
125 & ml & Milch \\
1 & Prise & Salz \\
1 & Prise & Zucker
}{Teig zusammen mischen und eine Studne ziehen lassen.

Pfannkuchen in einer gefetteten Pfanne ausbacken.
Nach Belieben mit Obst oder Gem\"use bestreuen.}

\rezept{Pizza-\"Ol-Teig}{20}{
Einfach}{
500 & g & Mehl\\
1 & TL & Salz \\
6 & EL & \"Ol \\
1 & Pck. & Hefe \\
300 & ml & Wasser
}{
Mehl, Salz und \"Ol vermischen.
1 W\"urfel Hefe in die Mitte br\"oseln.
Wasser dazugeben und durchkneten bis der Teig sich vom Rand l\"ost.

Tuch dr\"uberlegen und ruhen lassen.  }

\newpage

\rezept{Quarkbrot}{20}{
von Silvana}{
150 & g & Quark \\
500 & g & Mehl \\
2 & TL & Zucker \\
2 & TL & Salz \\
2 & Pck. & Backpulver \\
2 & & Eier
}{
Bei 180 Grad Ober- und Unterhitze ca. 45 min backen.  }

\rezept{Streusel}{20}{
1 Blech}{
200 & g & Mehl \\
100 & g & Zucker \\
100 & g & Butter \\
1 & Pck & Vanillezucker \\
1 & Prise & Salz
}{Zusammenmischen bis es kr\"umelt.}

\rezept{Streuselkekse}{20}{
1 Blech}{
200 & g & Mehl \\
100 & g & Zucker \\
100 & g & Butter \\
1 & Pck & Vanillezucker \\
1 & Prise & Salz \\
2-3 & EL & Pflaumenmuß
}{Streusel zusammenkneten und am Ende Marmelade hinzuf\"ugen, sodass eine feuchte, klebrige Masse entsteht.
Zu kleinen Kugeln formen und aufs Blech drücken, braucht kein Backpapier.
15-20 Minuten bei 170 Grad backen, bis die Kekse ein wenig gebräunt sind.}

\rezept{Zimtsterne}{20}{
vegan}{
300 & g & Puderzucker \\
2 & EL & Zimt \\
8 & EL & Wasser \\
1 & EL & Zitronensaft \\
150 & g & gehackte Mandeln \\
200 & g & gehackte Haseln\"usse \\
1 & EL & Orangenschale
}{
Vorsicht, der Teig klebt.
Etwa 10 Minuten backen.  }

\rezept{Schoko-Nuss Muffins}{20}{
etwa 16 Stück}{
150 & g & Haselnüsse, gerieben \\
150 & g & Karotten, gerieben \\
75 & g & Schokostückchen \\
30 & g & weiche Butter \\
30 & g & Mehl \\
120 & g & Zucker \\
2 & & Eier \\
1/2 & Pkg. & Backpulver \\
& & Salz
}{
Butter mit Salz und der Hälfte vom Zucker schaumig rühren. Dotter langsam untermengen.

Eiklar mit restlichem Zucker zu cremigem Schnee schlagen, unter die Dottermasse heben. Karotten und Haselnüsse untermengen.

Mehl, Backpulver und Schokoladestücke vermischen und unterheben.

Teig in die Muffinformen füllen und bei 160°C ca. 25 Minuten backen.}

\rezept{Apfel Muffins}{20}{
etwa 16 Stück}{
250 & g & Mehl \\
150 & g & Zucker \\
125 & g & Butter \\
2 & & Eier \\
1 & Pkg. & Vanillezucker \\
1/2 & Pkg. & Backpulver \\
250 & ml & Milch \\
2 & & Äpfel (z.B. Booskoop)
}{
Zutaten wie Mehl, Zucker, Margarine, Eier, Vanillezucker, Backpulver und Milch in eine große Schüssel geben und mit dem Mixer durchrühren.

Danach die geschälten, in kleine Stücke geschnittenen Äpfel hinzugeben und in den Teig mischen.

Die Papierförmchen jeweils zu 3/4 mit dem Teig füllen und auf ein Backblech stellen.

Im vorgeheizten Backofen bei 150°C Heißluft ca. 20-25 Minuten backen.}
