%\rezept{Titel}{20}{Untertitel}{150 & ml & Ei \\3 & Laib & Brot }{Ei aufs Brot spiegeln und mit Seffer und Pfalz bestreuen.}

\rezept{Champignon-Tomatenrahmso\ss{}e}{20}{
mit Basilikum}{
250 & g & Champignons \\
1 & & Zwiebel \\
2 & Zehen & Knoblauch \\
3 & & Tomaten \\
3 & EL & Tomatenmark \\
& etwas & Basilikum \\
150 & ml & Sahne \\
1 & TL & Gem\"usebr\"uhe \\
300 & g & Nudeln
}{
Die Nudeln normal kochen.

F\"ur die So\ss{}e die Zwiebeln und den Knoblauch klein schneiden, die Champignons putzen und in d\"unne Scheiben schneiden, die Tomaten klein w\"urfeln.

Zuerst die Zwieblen mit dem Knoblauch in einer Pfanne mit etwas \"Ol leicht anbraten, die Champignonscheiben dazugeben und mit anbraten.
Die Tomaten sowie das Tomatenmark zuf\"ugen, mit der Sahne aufgie\ss{}en und bei niedriger Temperatur k\"ocheln lassen.

Falls die So\ss{}e zu dickfl\"ussig ist Milch hinzugeben.

Nun das Basilikum hinzuf\"ugen und mit Pfeffer und Br\"uhe nach Geschmack w\"urzen.
}

\rezept{Curry}{20}{
Vegan, 90 min}{
2 & & Zwiebeln \\
3 & & Knoblauchzehen \\
150 & g & gesch\"alte \\
 & & Dosentomaten \\
2 & EL & Oliven\"ol \\
1 & EL & gelbe Currypaste \\
200 & g & Linsen \\
1-2 & TL & Kurkuma \\
400 & ml & Gem\"usebr\"uhe \\
200 & g & Blumenkohl \\
2 & & M\"ohren \\
3 & St & Staudensellerie \\
150 & g & Erbsen \\
150 & ml & Kokosmilch \\
80 & g & Margarine \\
1 & Prs & Salz \\
& & frischer Koriander
}{ Zwiebeln und Knoblauch abziehen und klein w\"urfeln.
Die gesch\"alten Tomaten hacken.
Oliven\"ol in breitem Topf erhitzen, Zwiebeln und Knoblauch darin 2-3 Minuten anschwitzen.
Die Currypaste unterr\"uhren, Linsen, Kurkuma und gehackte Tomaten dazugeben.
Das Gem\"use mit der Br\"uhe abl\"oschen und ca. 1 Stunde zugedeckt bei schwacher Hitze leise k\"ocheln lassen.

Inzwischen den Blumenkohl in kleine R\"oschen teilen, M\"ohren und Staudensellerie sch\"alen und in schr\"age Scheiben schneiden.
Das vorbereitete Gem\"use und die Erbsen ca. 10 Minuten vor Ende der Garzeit zu den Linsen geben, dann Kokosmilch und Margarine hinzuf\"ugen.

Das Linsencurry mit Salz abschmecken und mit Koriander garniert servieren.
Dazu passt ged\"ampfter Basmatireis.  }

\rezept{Els\"asser Flammkuchen}{20}{
Einfach und gut - f\"ur 4 Portionen}{
400 & g & Mehl (550) \\
20 & g & Hefe \\
1/4 & l & lauwarmes Wasser \\
3 & EL & \"Ol \\
125 & g & Speck \\
500 & g & Zwiebeln \\
1 & Becher & saure Sahne \\
1 & Becher & s\"u\ss{}e Sahne
}{
Hefe mit Wasser verr\"uhren, 10 Minuten stehen lassen.
Das Mehl in eine Sch\"ussel sieben, das \"ubrige Wasser, \"Ol und etwas Salz zugeben, zu einem glatten Teig verkneten und zugedeckt 30 Minuten an einem warmen Ort gehen lassen.

Speck in geine Streifen schneiden, Zwiebeln halbieren, in feine Scheiben hobeln.
S\"u\ss{}e und saure Sahne mit etwas Salz und Pfeffer verquirlen.

Teig vierteln.
Jedes St\"uck auf einem gro\ss{}en St\"uck Backpapier zu einer hauchd\"unnen rechteckigen Platte auswellen.
Sahnemischung darauf verstreichen und mit Zwiebeln und Speck bestreuen.

Bei Umluft 220 Grad etwa 10-15 Minuten goldbraun backen.  }

\newpage

\rezept{Linsenlasagne}{20}{vegetarisch}{
& & helle Lasagnebl\"atter \\
20 & g & Butter \\
50 & g & Parmesan \\
& & Bechamelso\ss{}e \\
& & \\
175 & g & getrocknete Linsen \\
& & oder \\
450 & g & Linsen aus der Dose \\
5 & & Tomaten \\
2 & & M\"ohren \\
2 & & Stangen Staudensellerie \\
100 & g & Porree \\
1 & & Zwiebel \\
1 & & Knoblauchzehe \\
1 & & Chilischote \\
3 & EL & Oliven\"ol \\
200 & ml & Gem\"usebr\"uhe \\
2 & EL & Tomatenmark \\
2 & EL & Rotwein \\
& & Thymian und Rosmarin \\
& & Pfeffer und Salz \\
& & Cayennepfeffer
}{
F\"ur die Bolognese:

Die Linsen eventuell einweichen und in reichlich Wasser gar kochen.
(Dauert mindestens 3 Stunden + 30 Minuten)
Dosenlinsen abtropfen lassen.

Die Tomaten kreuzweise einritzen, mit kochendem Wasser \"uberbr\"uhen, kalt absp\"ulen und h\"auten.
M\"ohren und Sellerie sch\"alen, Porree putzen.
Das Gem\"use in kleine W\"urfel schneiden.
Zwiebel und Knoblauch abziehen und klein hacken.
Chili absp\"ulen, entkernen, hacken.

Das \"Ol in einem Topf erhitzen und das Gem\"use darin kurz and\"unsten, etwas Br\"uhe dazugie\ss{}en.
Alles etwa 10-15 Minuten bei mittlerer Hitze kochen lassen.
Linsen abtropfen lassen und dazu geben.

Thymian und Rosmarin absp\"ulen, trocken tupfen, klein hacken und unter das Gem\"use heben.
Die Bolognese mit Rotwein, Salz, Pfeffer und Cayennepfeffer abschmecken.

Schichtanleitung:

Den Boden einer Auflaufform mit 3 EL Bechamelso\ss{}e bestreichen.
Die Form mit Lasagnebl\"attern auslegen.
Sollten sich die Bl\"atter \"uberlappen, etwas So\ss{}e dazwischengeben, damit die Doppelschicht nicht zu trocken wird.

Darauf eine etwa 1 cm dicke Schicht Bolognese gleichm\"a\ss{}ig verstreichen.
Einige Essl\"offel Bechamelso\ss{}e auf der Bolognese verteilen.
Danach wieder Lasagnebl\"atter darauflegen und so weiterschichten, bis alle Zutaten verbraucht sind.

Zum Schluss mit einer Schicht Lasagnebl\"atter und Bechamelso\ss{}e abschließen, dabei sollten die Lasagnebl\"atter gut mit So\ss{}e bedeckt sein, damit sie nicht austrocknen.

Mit etwa 50g Parmesan bestreuen und die Butter in Fl\"ockchen daraufsetzen.
Die Auflaufform auf der unteren Schiene bei 160 Grad Umluft 40-60 Minuten backen.
Wird die Kruste zu dunkel, mit einem St\"uck Backpapier abdecken.

Tip: Lasagne vorm Anschneiden etwa 10 Minuten ruhen lassen, dann ist sie leichter zu portionieren. }

\newpage

\rezept{Schmorgurken}{20}{auch mit Speck}
{1 & kg & Gurken\\
1/4 & l & Br\"uhe \\
40 & g & Speck \\
1 & & Zwiebel \\
1 & TL & Zucker \\
1 & EL & Mehl \\
 & & Dill\\
 & & Tomaten\\
 & & Kartoffeln als Beilage}{
Die Gurken sch\"alen, entkernen und in zweifingerbreite St\"ucke schneiden.
Den Speck und die Zwiebel w\"urfeln, dann anbraten.
Die Gurkenst\"uckchen hinzu und etwa 20 Minuten mitbraten, sodass sie glasig aussehen, aber noch ein wenig Biss haben.

Bei Bedarf kleingeschnittene Tomaten oder Kirschtomaten hinzu f\"ugen und kurz mitbraten.

Im Fett Zucker und Mehl br\"aunen, mit der Fleischbr\"uhe auff\"ullen, pikant mit Salz, Pfeffer nud Essig abschmecken.
\"Uber die geschmorten Gurken geben, kleingehackten Dill dazu, umr\"uhren und noch ein wenig durchziehen lassen.

Dazu Kartoffeln reichen.}

\rezept{Nudeln mediterran}{20}{von Seba}
{& & Nudeln \\
& & Sahne \\
& & Feta \\
& & Tomaten \\
& & getrocknete Tomaten \\
& & frische Champignons \\
& & Rucola \\
& & schwarze Oliven
}{
Nudeln mit einer Sahnesoße und siehe oben, Rucola zum Schluss unter die Soße mischen.
}

\rezept{Quark zu Kartoffeln}{20}{
a la Mama}{
& & fetter Quark \\
& & Tomaten \\
& & Paprika \\
& & Gurke \\
& & Schnittlauch \\
& & Salz \\
& & Pfeffer \\
& & Zucker}
{Gem\"use sehr klein w\"urfeln und mit den restlichen Zutaten vermischen.}

\newpage

\rezept{Volognese}{20}{Vegane Bolognese}{
500 & g & Tofu \\
250(?) & ml & Rotwein \\
8 & EL & Tomatenmark \\
200(?) & g & passierte Tomaten \\
1 & Bund & Basilikum \\
100 & ml & Oliven\"ol \\
2 & & Zwiebeln \\
4 & Zehen & Knoblauch \\
& & Zimt
}{
Den Tofu mit einer Gabel zerdr\"ucken und in Oliven\"ol scharf von allen Seiten anbraten.
Am Schluss die Zwiebeln und danach den Knoblauch hinzugeben und mit anbraten.
Pfeffern, salzen und ein wenig Zimt dazu geben.
Das Tomatenmark unterr\"uhren.
Wenn alles vermischt ist und der Tofu eine sch\"one Farbe angenommen hat mit dem Rotwein abl\"oschen.
Die passierten Tomaten hinzugeben und alles w\"urzen und einkochen lassen, evtl nachw\"urzen.
Das Basilikum hacken und mitkochen lassen.

Passt gut zu Nudeln, kann aber auch als Bolognese f\"ur die Linsenlasagne verwendet werden.
}
